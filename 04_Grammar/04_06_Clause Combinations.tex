\subsection{Clause Combinations}

\subsubsection{Complement Clauses}
Sumsgiwa forms complement clauses by adding a case-marking prefix onto the verb.
%TODO: example

\subsubsection{Adverbial Clauses}
Adverbial clauses use an adposition combined with a relative clause. The adposition, word expressing the function of the adverbial clause, and relativizer are shortened into one word.

%TODO: "At the time that…" shortened into one word
%TODO: example

\subsubsection{Relative Clauses} \label{04_06_03_Relative Clauses}
Relative clauses come after the head noun. They are introduced with the relativizer \textlangle \v{s}u\v{z}on\textrangle~and use a pronoun retention strategy for case recoverability. All clause elements can be relativized.
%TODO: example

\subsubsection{Coordination}

\paragraph{Conjunction}
There is no special morphosyntax for conjunction, with clauses simply juxtaposed next to each other. So, any two clauses in Sumsgiwa can be said to be conjoined.
%TODO: example

\paragraph{Disjunction}
Disjunction is expressed with the particle \textlangle zi\v{s}e\textrangle~inserted between the two clauses.
%TODO: example