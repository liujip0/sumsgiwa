\subsection{Adjective and Adverb Operations}

\subsubsection{Derivational Morphology}
Any adjective can take adverbial grammar markings, and vice versa.
%TODO: example

\subsubsection{Comparatives and Superlatives}
The comparative form of an adjective or adverb is formed by reduplicating the first syllable of the word stem.

\ex~[glstyle=nlevel]
\begingl
\glpreamble \SGwithRom{\v{z}u\v{z}u\v{s}o}
\endpreamble
\v{z}u\textasciitilde[{\Cmpr}\textasciitilde]@
\v{z}u\v{s}o[pretty]
\glft "prettier"
\endgl
\xe

There are two comparative constructions in Sumsgiwa.

The construction used for both adjectives and adverbs is splitting up the statement into two sentences. The adjective or adverb is in the comparative form, and the second, repeated, verb takes no grammatical markings.

%TODO: fix formatting and spacing
\ex~[glstyle=nlevel]
\begingl
\glpreamble \SGwithRom{siniñizu\v{z}u nibibi\v{s}en japode | ñizu nikuku\v{s}e ja\v{s}umzan}
\endpreamble
s-[{\Ind}-]@
ini-[{\Hab}-]@
ñizu[walk]@
-\v{z}u[-{\An}]
ni-[{\Hab}-]@
bi\textasciitilde[{\Cmpr}\textasciitilde]@
bi\v{s}en[quickly]
ja-[{\Nom}-]@
pode[cat]+
ñizu[walk]
ni-[{\Hab}-]@
ku\textasciitilde[{\Cmpr}\textasciitilde]@
ku\v{s}e[slowly]
ja-[{\Nom}-]@
\v{s}umzan[dog]
\glft "Cats walk quicker than dogs do." (lit. "Cats walk quickly, dogs walk slowly.")
\endgl
\xe

The second comparative construction only applies to adjectives. To compare two nouns, the comparative verb \textlangle kogu\textrangle~is used. It is only inflected for the gender of the comparee noun using the ergative suffixes. The comparee noun takes the ergative case and the standard noun takes the dative case and the comparative adjective in question.

\ex~[glstyle=nlevel]
\begingl
\glpreamble \SGwithRom{kogujo diki\v{z}e \v{s}osdandi \v{s}o\v{z}u\v{z}u\v{s}opo}
\endpreamble
kogu[{\Comp}]@
-jo[-{\Hg}]
di-[{\Erg}-]@
ki\v{z}e[boy]
\v{s}o-[{\Dat}-]@
sdandi[girl]
\v{s}o-[{\Dat}-]@
\v{z}u\textasciitilde[{\Cmpr}\textasciitilde]@
\v{z}u\v{s}o[pretty]@
-po[-{\Hg}]
\glft "The boy is prettier than the girl."
\endgl
\xe
