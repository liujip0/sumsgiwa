\subsection{Verb and Verb-Phrase Operations}

\subsubsection{Negation}
When verbs are negated, the first consonant of the verb stem changes from a voiced consonant to an unvoiced one, or vice versa, and the clause-level word order changes from VAP/VS (VSO) to AVP/SV (SVO). If the initial consonant is \textlangle y\textrangle~or \textlangle w\textrangle, it becomes \textlangle n\textrangle, \textlangle m\textrangle, or \textlangle ñ\textrangle, and vice versa. Which one the consonant becomes is unpredictable and must be memorized on a per-word basis.
%TODO: example

\subsubsection{Interrogatives}

\paragraph{Yes/No Questions}
Yes/no questions are marked by placing the question particle \textlangle \v{s}gezu\textrangle~at the beginning of the sentence. 
%TODO: example

\paragraph{Content Questions}
Content questions use the same interrogative marker as yes/no questions, \textlangle \v{s}gezu\textrangle, but also use the following question words in situ:

\begin{tabular}{|c|c|c|}
    \hline
    \Acrlong{nom} Noun & \textlangle sensu\textrangle & \TBstrut\\
    \hline
    \Acrlong{erg} Noun & \textlangle senyu\textrangle & \TBstrut\\
    \hline
    \Acrlong{acc} Noun & \textlangle senge\textrangle & \TBstrut\\
    \hline
    \Acrlong{dat} Noun & \textlangle senbi\textrangle & \TBstrut\\
    \hline
    \Acrlong{gen} Noun & \textlangle sen\v{z}o\textrangle & \TBstrut\\
    \hline
    Time Adverb & \textlangle \v{s}anbi\textrangle & \TBstrut\\
    \hline
    Location Adverb & \textlangle \v{s}anzim\textrangle & \TBstrut\\
    \hline
    Manner Adverb & \textlangle \v{s}an\v{z}e\textrangle & \TBstrut\\
    \hline
    Purpose/Reason Adverb & \textlangle \v{s}ansum\textrangle & \TBstrut\\
    \hline
\end{tabular}
%TODO: example

\subsubsection{Imperatives} \label{04_03_03_Imperatives}
Imperatives are in the \acrlong{deo} mood and have either the imperative particle \textlangle sesum\textrangle~or the negative imperative particle \textlangle dosika\textrangle~in front of the sentence. When the speaker is telling the hearer to do something, the agent is dropped and the patient goes into the nominative case.
%TODO: example

When both the agent and patient are present in the sentence, the speaker is telling the hearer to ask the agent to perform the action.
%TODO: example

\paragraph{Polite Imperatives}
Polite imperatives use the \acrlong{opt} mood instead of the \acrlong{deo} mood. Additionally, the agent is explicitly expressed using either the \glossfull{Spol} or \glossfull{Shum} pronouns.

\subsubsection{Causatives}
Causatives in Sumsgiwa are marked with the prefix \textlangle kaga\textrangle- on the fully-inflected verb. The tense/aspect inflections are still for the action itself rather than for causing the action. The causer takes the ergative case while the causee will take either the accusative or dative cases. The dative case is when the causee has little agency in the situation, while the accusative case is when they have some amount of agency.
%TODO: example

\subsubsection{Possessor Raising}
Intransitive verbs with possessed subjects can be expressed as transitive verbs with the possessee as the agent and the possessor as the patient.
%TODO: example

\subsubsection{Argument Omission}
Any argument of a verb can be omitted as long as the omitted noun is clear from context.
%TODO: example

\subsubsection{Verb Compounding}

\paragraph{Noun Incorporation}

\subparagraph{Subject/Agent Incorporation}
Subjects and agents are incorporated by moving their uninflected forms before the verb stem. Any prefixes that go on the verb will move to before the incorporated noun.
%TODO: example

\subparagraph{Patient Incorporation}
Patients are incorporated by moving their uninflected forms after the verb stem. Any suffixes the verb may take will go after the incorporated patient.
%TODO: example

\paragraph{Verb-Verb Incorporation}
Verbs are compounded by juxtaposition, with inflections applied to the new combined word.
%TODO: example

\subsubsection{Nominalization}

\paragraph{Action Nominalization}
Action nominalization is achieved by adding case markings onto the uninflected form of the verb.
%TODO: example

\paragraph{Agent/Subject Nominalization}
Sumsgiwa nominalizes the typical agent or subject of a verb by adding the prefix \textlangle juzi\textrangle- to its uninflected form.
%TODO: example

\paragraph{Patient Nominalization}
The typical patients of verbs are nominalized by adding the suffix -\textlangle sokuñ\textrangle~to the verbs' uninflected forms.
%TODO: example

\paragraph{Instrument Nominalization}
Sumsgiwa nominalizes the typical instruments of a verb by adding the suffix -\textlangle sodise\textrangle~to its uninflected form.
%TODO: example

\paragraph{Location Nominalization}
Sumsgiwa nominalizes the typical locations where a verb happens by adding the suffix -\textlangle gomsa\textrangle~to its uninflected form.
%TODO: example

\paragraph{Product Nominalization}
Sumsgiwa nominalizes the typical products of the action represented by a verb by adding the suffix -\textlangle sigo\textrangle~to its uninflected form.
%TODO: example
