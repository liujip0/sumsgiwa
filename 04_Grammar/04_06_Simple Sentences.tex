\subsection{Simple Sentences}

\subsubsection{Simple Declarative Sentences}
Simple declarative sentences in Sumsgiwa are expressed with VAP/VS word order. Transitive verbs agree with both the agent and patient, and intransitive verbs agree with the subject [See \fullref{04_01_02_04_Agreement with Nouns}].
%TODO: example

\subsubsection{Predicate Nominals} \label{04_05_02_Predicate Nominals}
Predicate nominals use a defective \glossfull{cop} verb, \textlangle se\textrangle, that inflects for mood but not aspect or gender. These constructions encompass proper inclusion, where something is asserted to be among the class specified by the nominal predicate, and equative clauses, where two things are asserted to be the same.
%TODO: example

\subsubsection{Predicate Adjectives}
Predicate adjectives in Sumsgiwa consist of the noun and properly inflected adjective juxtaposed with each other, with no copular verbs or particles.
%TODO: example

\subsubsection{Predicate Locatives}
Predicate locative constructions use the locative adposition, \textlangle \v{s}im\textrangle, as a defective verb, inflecting it for mood but not for aspect or gender.
%TODO: example

\subsubsection{Existentials}
Existentials use the same copular verb as predicate nominals, \textlangle se\textrangle, and also only inflect it for mood.
%TODO: example

\subsubsection{Possessive Clauses}
Possessive clauses use the same verb as predicate locatives, \textlangle \v{s}im\textrangle, also only inflected for mood. The possessor is placed in the genitive case.
%TODO: example