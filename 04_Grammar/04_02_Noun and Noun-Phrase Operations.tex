\subsection{Noun and Noun-Phrase Operations}

\subsubsection{Noun-Noun Compounding}
Nouns are compounded by juxtaposition, with the new combined form inflected as one word.
%TODO: example

\subsubsection{Denominalization}
There are two ways nouns can be verbalized, with different meanings. In both cases, morphological changes are applied to the uninflected forms of the nouns.

The prefix \textlangle do\textrangle - converts the noun into a verb meaning "to become [noun]."
%TODO: example

The prefix \textlangle ke\textrangle - changes the noun into a verb that means "to have [noun]."
%TODO: example

\subsubsection{Adjectivization}
A noun can take adjectival case and gender markings to become an adjective meaning "similar to [noun]."
%TODO: example

\subsubsection{Determiners}

\paragraph{Demonstratives}
Demonstratives agree with the nouns they modify in gender, and are split into 3 categories. They go after the nouns they modify.
\begin{description}
    \item[\Glossfull{prox}] Close to the speaker
    \item[\Glossfull{med}] Close to the listener
    \item[\Glossfull{dist}] Far from both the speaker and the listener
\end{description}
\begin{tabular}{|l|c|c|c|}
    \hline
    & \Glossfull{prox} &
        \Glossfull{med} &
        \Glossfull{dist} \TBstrut\\
    \hline

    \multirow{2}{*}{\Glossfull{hg}} &
        \textlangle ke\v{s}i\textrangle &
        \textlangle zi\v{s}go\textrangle &
        \textlangle simsbe\textrangle \Tstrut\\
        & & & \Bstrut\\
    \hline

    \multirow{2}{*}{\Glossfull{an}} &
        \textlangle gunzin\textrangle &
        \textlangle sgaku\textrangle &
        \textlangle densi\textrangle \Tstrut\\
        & & & \Bstrut\\
    \hline

    \multirow{2}{*}{\Glossfull{inan}} &
        \textlangle dazo\textrangle &
        \textlangle yezi\textrangle &
        \textlangle \v{s}osu\textrangle \Tstrut\\
        & & & \Bstrut\\
    \hline
\end{tabular}

Sumsgiwa does not have demonstrative pronouns, so it uses a demonstrative adjective and a generic noun to specify the same meaning.
