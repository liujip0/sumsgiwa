\subsection{Noun and Noun-Phrase Operations}

\subsubsection{Noun-Noun Compounding}
Nouns are compounded by juxtaposition, with the new combined form inflected as one word.
%TODO: example

\subsubsection{Denominalization}
There are two ways nouns can be verbalized, with different meanings. In both cases, morphological changes are applied to the uninflected forms of the nouns.

The prefix \textlangle do\textrangle - converts the noun into a verb meaning "to become [noun]."

\ex~[glstyle=nlevel]
\begingl
\glpreamble \SGwithRom{do\v{c}uzipo \v{c}em}
\endpreamble
Ø-[{\Ind}-]@
Ø-[{\Pfv}-]@
do-[{\Vbz}-]@
\v{c}uzi[man]@
-po[-{\Hg}]
\v{c}em[{\T}.{\Hg}.{\Nom}]
\glft "He became a man." (e.g. "He transitioned.")
\endgl
\xe

The prefix \textlangle ke\textrangle - changes the noun into a verb that means "to have [noun]."

\ex~[glstyle=nlevel]
\begingl
\glpreamble \SGwithRom{sakebi\v{s}epo \v{c}em}
\endpreamble
s-[{\Ind}-]@
a-[{\Prog}-]@
ke-[{\Vbz}-]@
bi\v{s}e[table]@
-po[-{\Hg}]
\v{c}em[{\T}.{\Hg}.{\Nom}]
\glft "They own tables."
\endgl
\xe

\subsubsection{Adjectivization}
A noun can take adjectival case and gender markings to become an adjective meaning "similar to [noun]."

\ex~[glstyle=nlevel]
\begingl
\glpreamble \SGwithRom{jakengo \.{z}e\v{s}i \v{c}adezu\v{z}u}
\endpreamble
ja-[{\Nom}-]@
kengo[action]
\.{z}e\v{s}i[{\Shum}.{\Hg}.{\Gen}]
\v{c}a-[{\Nom}.{\Adjz}-]@
dezu[jaguar]@
-\v{z}u[-{\An}.{\Adjz}]
\glft "Your actions are jaguar-like!"
\endgl
\xe

\subsubsection{Determiners}

\paragraph{Demonstratives}
Demonstratives agree with the nouns they modify in gender, and are split into 3 categories. They go after the nouns they modify.
\begin{description}
  \item[\Glossfull{prox}] Close to the speaker
  \item[\Glossfull{med}] Close to the listener
  \item[\Glossfull{dist}] Far from both the speaker and the listener
\end{description}
{
  \newcommand{\TableRow}[4]{
    \multirow{2}{*}{\Glossfull{#1}} &
    \textlangle #2\textrangle &
    \textlangle #3\textrangle &
    \textlangle #4\textrangle \Tstrut\\
    & \SG{#2} & \SG{#3} & \SG{#4} \Bstrut\\
    \hline
  }

  \begin{tabular}{|l|c|c|c|}
    \hline
    & \Glossfull{prox} &
    \Glossfull{med} &
    \Glossfull{dist} \TBstrut\\
    \hline

    \TableRow{hg}{ke\v{s}i}{zi\v{s}go}{simsbe}

    \TableRow{an}{gunzin}{sgaku}{densi}

    \TableRow{inan}{dazo}{yezi}{\v{s}osu}
  \end{tabular}
}

Sumsgiwa does not have demonstrative pronouns, so it uses a demonstrative adjective and a generic noun to specify the same meaning.

\paragraph{Distributive Determiners}

Distributive determiners agree with the nouns they modify in gender.

See \fullref{05_02_D-Quantifiers}.

\paragraph{Determiners of Difference}

Determiners of difference agree with the nouns they modify in gender.

{
  \newcommand{\TableRow}[2]{
    \multirow{2}{*}{\Glossfull{#1}} &
    \textlangle #2\textrangle \Tstrut\\
    & \SG{#2} \Bstrut\\
    \hline
  }

  \begin{tabular}{|l|c|}
    \hline
    & "another/other" \TBstrut\\
    \hline

    \TableRow{hg}{dim\v{z}i}

    \TableRow{an}{tu\v{z}e}

    \TableRow{inan}{\v{z}osa}
  \end{tabular}
}
