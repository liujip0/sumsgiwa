\subsection{Parts of Speech}

{
  \newcommand{\TableRowPrefix}[2]{
    \Glossfull{#1} &
    \textlangle #2\textrangle - & \SG{#2} \TBstrut\\
    \hline
  }
  \newcommand{\TableRowSuffix}[2]{
    \Glossfull{#1} &
    -\textlangle #2\textrangle & \SG{#2} \TBstrut\\
    \hline
  }
  \newcommand{\TableRowInfix}[2]{
    \Glossfull{#1} &
    -\textlangle #2\textrangle - & \SG{#2} \TBstrut\\
    \hline
  }

  \subsubsection{Nouns}

  \paragraph{Noun Phrases}
  Noun phrases are structured as follows:

  {[}Adposition{]}\\
  {[}Ordinal{]}\\
  Case-NOUN STEM-{[}Augmentative/Diminutive{]}\\
  {[}Adjective(s){]}\\
  {[}Comparative{}]\\
  {[}Cardinal/Quantifier{]}\\
  {[}Demonstrative{]}\\
  {[}Genitive{]}\\
  {[}Relative Phrase{]}

  {[]} = optional part of noun phrase
  %TODO: example

  \paragraph{Noun Cases}
  There are 5 cases in Sumsgiwa indicated using prefixes:
  \begin{description}
    \item[\Glossfull{nom}] \Gls{subject} of an intransitive verb
    \item[\Glossfull{erg}] \Gls{agent} of a transitive verb
    \item[\Glossfull{acc}] \Gls{patient} of a transitive verb
    \item[\Glossfull{dat}] Indirect object of a verb
    \item[\Glossfull{gen}] Modifier of another noun
  \end{description}
  \begin{tabular}{|l|c|c|}
    \hline
    \TableRowPrefix{nom}{ja}
    \TableRowPrefix{erg}{di}
    \TableRowPrefix{acc}{\v{s}a}
    \TableRowPrefix{dat}{\v{s}o}
    \TableRowPrefix{gen}{ne}
  \end{tabular}

  \paragraph{Gender}
  There are 3 genders, and nouns are categorized purely through semantics, rather than through phonology or spelling.
  \begin{description}
    \item[\Glossfull{hg}] Humans, body parts, nouns relating to language
    \item[\Glossfull{an}] Animals, thoughts
    \item[\Glossfull{inan}] Objects, plants, fungi
  \end{description}
  Both adjectives and verbs must agree with nouns in gender.

  \subsubsection{Verbs}

  \paragraph{Verb Phrases}
  Verb phrases are structured as follows:

  {[}Question Particle{]}\\
  Mood-Aspect-VERB STEM-Gender of S-Gender of A-Gender of P\\
  {[}Adverb(s){]}

  {[]} = optional part of verb phrase
  %TODO: example

  \paragraph{Mood}
  Verbs have 4 moods marked using prefixes:
  \begin{description}
    \item[\Glossfull{ind}] Default, all \gls{realis} statements
    \item[\Glossfull{opt}] Wishes, hopes, and desires
    \item[\Glossfull{deo}] How things "ought" to be
    \item[\Glossfull{sjv}] All other \gls{irrealis} statements
  \end{description}
  \begin{tabular}{|l|c|c|}
    \hline
    \Glossfull{ind} &
    Ø- or \textlangle s\textrangle - & \SG{s} \TBstrut\\
    \hline
    \Glossfull{opt} &
    \textlangle \v{s}d\textrangle - & \SG{\v{s}d} \TBstrut\\
    \hline
    \Glossfull{deo} &
    \textlangle \v{s}u\v{c}\textrangle - & \SG{\v{s}u\v{c}} \TBstrut\\
    \hline
    \Glossfull{sjv} &
    \textlangle y\textrangle - & \SG{y} \TBstrut\\
    \hline
  \end{tabular}

  Imperative statements use the \acrlong{deo} mood. [See \fullref{04_03_03_Imperatives}]

  \paragraph{Aspect}
  Verbs have 4 aspects marked with prefixes:
  \begin{description}
    \item[\Glossfull{pfv}] Complete action as a single event in time
    \item[\Glossfull{hab}] Habitual actions (repetition over multiple occasions)
    \item[\Glossfull{iter}] Repeated actions (repetition at a single occasion)
    \item[\Glossfull{prog}] Action in progress at a specific time (incl. continuous)
  \end{description}
  \begin{tabular}{|l|c|c|}
    \hline
    \Glossfull{pfv} &
    -Ø- or -\textlangle e\textrangle - & \SG{se} \TBstrut\\
    \hline
    \Glossfull{hab} &
    -\textlangle ini\textrangle - & \SG{sini} \TBstrut\\
    \hline
    \Glossfull{iter} &
    -\textlangle i\textrangle - & \SG{si} \TBstrut\\
    \hline
    \Glossfull{prog} &
    -\textlangle a\textrangle - & \SG{sa} \TBstrut\\
    \hline
  \end{tabular}

  \paragraph{Agreement with Nouns} \label{04_01_02_04_Agreement with Nouns}
  Verbs agree with their subject, agent, and patient arguments in gender.

  \subparagraph{Subject Gender}
  \begin{tabular}{|l|c|c|}
    \hline
    \TableRowSuffix{hg}{po}
    \TableRowSuffix{an}{\v{z}u}
    \TableRowSuffix{inan}{\v{s}bi}
  \end{tabular}

  \subparagraph{Agent Gender}
  \begin{tabular}{|l|c|c|}
    \hline
    \TableRowInfix{hg}{jo}
    \TableRowInfix{an}{\v{c}e}
    \TableRowInfix{an}{sba}
  \end{tabular}

  \subparagraph{Patient Gender}
  \begin{tabular}{|l|c|c|}
    \hline
    \TableRowSuffix{hg}{pon}
    \TableRowSuffix{an}{\v{s}um}
    \TableRowSuffix{inan}{\v{z}o}
  \end{tabular}

  \subsubsection{Adjectives}
  Adjectives are structured as follows:

  Case-ADJECTIVE STEM-Gender
  %TODO: example

  \paragraph{Agreement with Nouns}
  Adjectives agree with the nouns they modify in case and gender.

  \subparagraph{Case}
  \begin{tabular}{|l|c|c|}
    \hline
    \TableRowPrefix{nom}{\v{c}a}
    \TableRowPrefix{erg}{ti}
    \TableRowPrefix{acc}{\v{s}a}
    \TableRowPrefix{dat}{\v{s}o}
    \TableRowPrefix{gen}{ne}
  \end{tabular}

  \subparagraph{Gender}
  \begin{tabular}{|l|c|c|}
    \hline
    \TableRowSuffix{hg}{po}
    \TableRowSuffix{an}{\v{z}u}
    \TableRowSuffix{inan}{\v{s}bi}
  \end{tabular}

  \subsubsection{Adverbs}
  Adverb structure:

  Aspect-ADVERB STEM

  \textit{or}

  ADVERB STEM-Gender
  %TODO: example

  \paragraph{Agreement with Verbs}
  Adverbs follow the verbs they modify and agree with them in aspect.

  \subparagraph{Aspect}
  \begin{tabular}{|l|c|c|}
    \hline
    \Glossfull{pfv} &
    Ø- & \TBstrut\\
    \hline
    \TableRowPrefix{hab}{ni}
    \TableRowPrefix{iter}{neye}
    \TableRowPrefix{prog}{na}
  \end{tabular}

  \paragraph{Agreement with Adjectives}
  Adverbs follow the adjectives they modify and agree with them in gender.

  \subparagraph{Gender}
  \begin{tabular}{|l|c|c|}
    \hline
    \TableRowSuffix{hg}{po}
    \TableRowSuffix{an}{\v{s}u}
    \TableRowSuffix{inan}{\v{s}be}
  \end{tabular}
}

\subsubsection{Pronouns}
There are different pronoun forms for each of the 5 cases. Pronouns in Sumsgiwa distinguish between 1st person exclusive and 1st person inclusive, and there are 3 politeness levels for 2nd person pronouns. In addition, there are 3 sets of pronouns for each of the 3 genders.
\begin{description}
  \item[\Glossfull{Fex}] 1st person, excluding the hearer
  \item[\Glossfull{Fin}] 1st person, including the hearer
  \item[\Glossfull{Spol}] 2nd person, used in formal or neutral circumstances
  \item[\Glossfull{Shum}] 2nd person, used when speaking to figures of authority, elders, etc.
  \item[\Glossfull{Sfam}] 2nd person, used when speaking to friends and/or family members
  \item[\Glossfull{T}] 3rd person
\end{description}

{
  \newcommand{\TableRow}[6]{
    \multirow{2}{5em}{\Glossfull{#1}} &
    \textlangle #2\textrangle &
    \textlangle #3\textrangle &
    \textlangle #4\textrangle &
    \textlangle #5\textrangle &
    \textlangle #6\textrangle \Tstrut\\
    & \SG{#2} & \SG{#3} & \SG{#4} & \SG{#5} & \SG{#6} \Bstrut\\
    \hline
  }
  \newcommand{\TableRowTall}[6]{
    \multirow{4}{5em}{\Glossfull{#1}} &
    \textlangle #2\textrangle &
    \textlangle #3\textrangle &
    \textlangle #4\textrangle &
    \textlangle #5\textrangle &
    \textlangle #6\textrangle \Tstrut\\
    & \SG{#2} & \SG{#3} & \SG{#4} & \SG{#5} & \SG{#6} \\
    & & & & & \\
    & & & & & \Bstrut\\
    \hline
  }
  \newcommand{\TableHead}{\hline
    & \Glossfull{nom} &
    \Glossfull{erg} &
    \Glossfull{acc} &
    \Glossfull{dat} &
    \Glossfull{gen} \TBstrut\\
  \hline}

  \paragraph{Human}

  \begin{tabular}{|m{5em}|m{5em}|m{4em}|m{5em}|m{4em}|m{4em}|}
    \TableHead

    \TableRowTall{Fex}{yem}{ya}{yu}{ye\.{z}o}{yañi}

    \TableRowTall{Fin}{\v{z}em}{\v{z}a}{\v{z}u}{\v{z}e\.{z}o}{\v{z}iñi}

    \TableRowTall{Spol}
    {ji\v{s}bem}{ji\v{s}ba}{ji\v{s}bu}{ji\.{z}o}{ji\v{s}i}

    \TableRowTall{Shum}
    {\.{z}e\v{s}bem}{\.{z}e\v{s}ba}{\.{z}e\v{s}bu}{\.{z}e\v{z}o}{\.{z}e\v{s}i}

    \TableRowTall{Sfam}{\v{s}bem}{\v{s}ba}{\v{s}bu}{\v{s}be\.{z}o}{\v{s}biñi}

    \TableRow{T}{\v{c}em}{\v{c}a}{\v{c}u}{\v{c}u\.{z}o}{\v{c}añi}
  \end{tabular}

  \paragraph{Animate}

  \begin{tabular}{|m{5em}|m{5em}|m{4em}|m{5em}|m{4em}|m{4em}|}
    \TableHead

    \TableRow{S}{consi}{consan}{conso}{sonsu\v{c}e}{consu\v{c}i}

    \TableRow{T}{si}{san}{so}{su\v{c}e}{su\v{c}i}
  \end{tabular}

  \paragraph{Inanimate}
  \begin{tabular}{|m{5em}|m{5em}|m{4em}|m{5em}|m{4em}|m{4em}|}
    \TableHead

    \TableRow{T}{yo}{zin}{je}{sayo}{zeyo}
  \end{tabular}
}

\subsubsection{Adpositions}
Sumsgiwa uses prepositions that take uninflected nouns. The prepositional phrases follow the verb. However, some functions usually covered by adpositions in other languages are covered by the \acrlong{dat} case.
%TODO: example
