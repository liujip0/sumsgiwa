\subsection{Parts of Speech}

\subsubsection{Nouns}

\paragraph{Noun Phrases}
Noun phrases are structured as follows:

{[}Adposition{]}\\
{[}Ordinal{]}\\
Case-NOUN STEM-{[}Augmentative/Diminutive{]}\\
{[}Adjective(s){]}\\
{[}Comparative{}]\\
{[}Cardinal/Quantifier{]}\\
{[}Demonstrative{]}\\
{[}Genitive{]}\\
{[}Relative Phrase{]}

{[]} = optional part of noun phrase
%TODO: example

\paragraph{Noun Cases}
There are 5 cases in Sumsgiwa indicated using prefixes:
\begin{description}
    \item[\Glossfull{nom}] \Gls{subject} of an intransitive verb
    \item[\Glossfull{erg}] \Gls{agent} of a transitive verb
    \item[\Glossfull{acc}] \Gls{patient} of a transitive verb 
    \item[\Glossfull{dat}] Indirect object of a verb
    \item[\Glossfull{gen}] Modifier of another noun 
\end{description}
\begin{tabular}{|l|c|c|}
    \hline
    \Glossfull{nom} &
        \textlangle ja\textrangle - & \TBstrut\\
    \hline
    \Glossfull{erg} &
        \textlangle di\textrangle - & \TBstrut\\
    \hline
    \Glossfull{acc} &
        \textlangle ša\textrangle - & \TBstrut\\
    \hline
    \Glossfull{dat} &
        \textlangle šo\textrangle - & \TBstrut\\
    \hline
    \Glossfull{gen} &
        \textlangle ne\textrangle - & \TBstrut\\
    \hline
\end{tabular}

\paragraph{Gender}
There are 3 genders, and nouns are categorized purely through semantics, rather than through phonology or spelling.
\begin{description}
    \item[\Glossfull{hg}] Humans, body parts, nouns relating to language
    \item[\Glossfull{an}] Animals, thoughts
    \item[\Glossfull{inan}] Objects, plants, fungi 
\end{description}
Both adjectives and verbs must agree with nouns in gender.

\subsubsection{Verbs}

\paragraph{Verb Phrases}
Verb phrases are structured as follows:

{[}Question Particle{]}\\
Mood-Aspect-VERB STEM-Gender of S-Gender of A-Gender of P\\
{[}Adverb(s){]}

{[]} = optional part of verb phrase
%TODO: example

\paragraph{Mood}
Verbs have 4 moods marked using prefixes:
\begin{description}
    \item[\Glossfull{ind}] Default, all \gls{realis} statements
    \item[\Glossfull{opt}] Wishes, hopes, and desires
    \item[\Glossfull{deo}] How things "ought" to be
    \item[\Glossfull{sjv}] All other \gls{irrealis} statements 
\end{description}
\begin{tabular}{|l|c|c|}
    \hline
    \Glossfull{ind} &
        Ø- or \textlangle s\textrangle - & \TBstrut\\
    \hline
    \Glossfull{opt} &
        \textlangle šd\textrangle - & \TBstrut\\
    \hline
    \Glossfull{deo} &
        \textlangle šuč\textrangle - & \TBstrut\\
    \hline
    \Glossfull{sjv} &
        \textlangle y\textrangle - & \TBstrut\\
    \hline
\end{tabular}

Imperative statements use the \acrlong{deo} mood. [See \fullref{04_03_03_Imperatives}]

\paragraph{Aspect}
Verbs have 4 aspects marked with prefixes:
\begin{description}
    \item[\Glossfull{pfv}] Complete action as a single event in time
    \item[\Glossfull{hab}] Habitual actions (repetition over multiple occasions)
    \item[\Glossfull{iter}] Repeated actions (repetition at a single occasion)
    \item[\Glossfull{prog}] Action in progress at a specific time (incl. continuous)
\end{description}
\begin{tabular}{|l|c|c|}
    \hline
    \Glossfull{pfv} &
        -Ø- or -\textlangle e\textrangle - & \TBstrut\\
    \hline
    \Glossfull{hab} &
        -\textlangle ini\textrangle - & \TBstrut\\
    \hline
    \Glossfull{iter} &
        -\textlangle i\textrangle - & \TBstrut\\
    \hline
    \Glossfull{prog} &
        -\textlangle a\textrangle - & \TBstrut\\
    \hline
\end{tabular}

\paragraph{Agreement with Nouns} \label{04_01_02_04_Agreement with Nouns}
Verbs agree with their subject, agent, and patient arguments in gender.

\subparagraph{Subject Gender}
\begin{tabular}{|l|c|c|}
    \hline
    \Glossfull{hg} &
        -\textlangle po\textrangle &\TBstrut\\
    \hline
    \Glossfull{an} &
        -\textlangle žu\textrangle &\TBstrut\\
    \hline
    \Glossfull{inan} &
        -\textlangle šbi\textrangle &\TBstrut\\
    \hline
\end{tabular}

\subparagraph{Agent Gender}
\begin{tabular}{|l|c|c|}
    \hline
    \Glossfull{hg} &
        -\textlangle jo\textrangle - &\TBstrut\\
    \hline
    \Glossfull{an} &
        -\textlangle če\textrangle - &\TBstrut\\
    \hline
    \Glossfull{inan} &
        -\textlangle sba\textrangle - &\TBstrut\\
    \hline
\end{tabular}

\subparagraph{Patient Gender}
\begin{tabular}{|l|c|c|}
    \hline
    \Glossfull{hg} &
        -\textlangle pon\textrangle &\TBstrut\\
    \hline
    \Glossfull{an} &
        -\textlangle šum\textrangle &\TBstrut\\
    \hline
    \Glossfull{inan} &
        -\textlangle žo\textrangle &\TBstrut\\
    \hline
\end{tabular}

\subsubsection{Adjectives}
Adjectives are structured as follows:

Case-ADJECTIVE STEM-Gender
%TODO: example

\paragraph{Agreement with Nouns}
Adjectives agree with the nouns they modify in case and gender.

\subparagraph{Case}
\begin{tabular}{|l|c|c|}
    \hline
    \Glossfull{nom} &
        \textlangle ča\textrangle - & \TBstrut\\
    \hline
    \Glossfull{erg} &
        \textlangle ti\textrangle - & \TBstrut\\
    \hline
    \Glossfull{acc} &
        \textlangle ša\textrangle - & \TBstrut\\
    \hline
    \Glossfull{dat} &
        \textlangle šo\textrangle - & \TBstrut\\
    \hline
    \Glossfull{gen} &
        \textlangle ne\textrangle - & \TBstrut\\
    \hline
\end{tabular}

\subparagraph{Gender}
\begin{tabular}{|l|c|c|}
    \hline
    \Glossfull{hg} &
        -\textlangle po\textrangle &\TBstrut\\
    \hline
    \Glossfull{an} &
        -\textlangle žu\textrangle &\TBstrut\\
    \hline
    \Glossfull{inan} &
        -\textlangle šbi\textrangle &\TBstrut\\
    \hline
\end{tabular}

\subsubsection{Adverbs}
Adverb structure:

Aspect-ADVERB STEM

\textit{or}

ADVERB STEM-Gender
%TODO: example

\paragraph{Agreement with Verbs}
Adverbs follow the verbs they modify and agree with them in aspect.

\subparagraph{Aspect}
\begin{tabular}{|l|c|c|}
    \hline
    \Glossfull{pfv} &
        Ø- & \TBstrut\\
    \hline
    \Glossfull{hab} &
        \textlangle ni\textrangle - & \TBstrut\\
    \hline
    \Glossfull{iter} &
        \textlangle neye\textrangle - & \TBstrut\\
    \hline
    \Glossfull{prog} &
        \textlangle na\textrangle - & \TBstrut\\
    \hline
\end{tabular}

\paragraph{Agreement with Adjectives}
Adverbs follow the adjectives they modify and agree with them in gender.

\subparagraph{Gender}
\begin{tabular}{|l|c|c|}
    \hline
    \Glossfull{hg} &
        -\textlangle po\textrangle &\TBstrut\\
    \hline
    \Glossfull{an} &
        -\textlangle šu\textrangle &\TBstrut\\
    \hline
    \Glossfull{inan} &
        -\textlangle šbe\textrangle &\TBstrut\\
    \hline
\end{tabular}

\subsubsection{Pronouns}
There are different pronoun forms for each of the 5 cases. Pronouns in Sumsgiwa distinguish between 1st person exclusive and 1st person inclusive, and there are 3 politeness levels for 2nd person pronouns. In addition, there are 3 sets of pronouns for each of the 3 genders.
\begin{description}
    \item[\Glossfull{Fex}] 1st person, excluding the hearer
    \item[\Glossfull{Fin}] 1st person, including the hearer
    \item[\Glossfull{Spol}] 2nd person, used in formal or neutral circumstances
    \item[\Glossfull{Shum}] 2nd person, used when speaking to figures of authority, elders, etc.
    \item[\Glossfull{Sfam}] 2nd person, used when speaking to friends and/or family members
    \item[\Glossfull{T}] 3rd person 
\end{description}

\paragraph{Human}
\begin{tabular}{|m{5em}|m{5em}|m{4em}|m{5em}|m{4em}|m{4em}|}
    \hline
    & \Glossfull{nom} &
        \Glossfull{erg} &
        \Glossfull{acc} &
        \Glossfull{dat} &
        \Glossfull{gen} \TBstrut\\
    \hline

    \multirow{4}{5em}{\Glossfull{Fex}} &
        \textlangle yem\textrangle &
        \textlangle ya\textrangle &
        \textlangle yu\textrangle &
        \textlangle yeżo\textrangle &
        \textlangle yañi\textrangle \Tstrut\\
        & & & & & \\
        & & & & & \\
        & & & & & \Bstrut\\
    \hline

    \multirow{4}{5em}{\Glossfull{Fin}} &
        \textlangle žem\textrangle &
        \textlangle ža\textrangle &
        \textlangle žu\textrangle &
        \textlangle žeżo\textrangle &
        \textlangle žiñi\textrangle \Tstrut\\
        & & & & & \\
        & & & & & \\
        & & & & & \Bstrut\\
    \hline

    \multirow{4}{5em}{\Glossfull{Spol}} &
        \textlangle jišbem\textrangle &
        \textlangle jišba\textrangle &
        \textlangle jišbu\textrangle &
        \textlangle jiżo\textrangle &
        \textlangle jiši\textrangle \Tstrut\\
        & & & & & \\
        & & & & & \\
        & & & & & \Bstrut\\
    \hline

    \multirow{4}{5em}{\Glossfull{Shum}} &
        \textlangle żešbem\textrangle &
        \textlangle żešba\textrangle &
        \textlangle żešbu\textrangle &
        \textlangle żežo\textrangle &
        \textlangle żeši\textrangle \Tstrut\\
        & & & & & \\
        & & & & & \\
        & & & & & \Bstrut\\
    \hline

    \multirow{4}{5em}{\Glossfull{Sfam}} &
        \textlangle šbem\textrangle &
        \textlangle šba\textrangle &
        \textlangle šbu\textrangle &
        \textlangle šbeżo\textrangle &
        \textlangle šbiñi\textrangle \Tstrut\\
        & & & & & \\
        & & & & & \\
        & & & & & \Bstrut\\
    \hline

    \multirow{2}{5em}{\Glossfull{T}} &
        \textlangle čem\textrangle &
        \textlangle ča\textrangle &
        \textlangle ču\textrangle &
        \textlangle čużo\textrangle &
        \textlangle čañi\textrangle \Tstrut\\
        & & & & & \Bstrut\\
    \hline
\end{tabular}

\paragraph{Animate}
\begin{tabular}{|m{5em}|m{5em}|m{4em}|m{5em}|m{4em}|m{4em}|}
    \hline
    & \Glossfull{nom} &
        \Glossfull{erg} &
        \Glossfull{acc} &
        \Glossfull{dat} &
        \Glossfull{gen} \TBstrut\\
    \hline

    \multirow{2}{5em}{\Glossfull{S}} &
        \textlangle consi\textrangle &
        \textlangle consan\textrangle &
        \textlangle conso\textrangle &
        \textlangle sonsuče\textrangle &
        \textlangle consuči\textrangle \Tstrut\\
        & & & & & \Bstrut\\
    \hline

    \multirow{2}{5em}{\Glossfull{T}} &
        \textlangle si\textrangle &
        \textlangle san\textrangle &
        \textlangle so\textrangle &
        \textlangle suče\textrangle &
        \textlangle suči\textrangle \Tstrut\\
        & & & & & \Bstrut\\
    \hline
\end{tabular}

\paragraph{Inanimate}
\begin{tabular}{|m{5em}|m{5em}|m{4em}|m{5em}|m{4em}|m{4em}|}
    \hline
    & \Glossfull{nom} &
        \Glossfull{erg} &
        \Glossfull{acc} &
        \Glossfull{dat} &
        \Glossfull{gen} \TBstrut\\
    \hline

    \multirow{2}{5em}{\Glossfull{T}} &
        \textlangle yo\textrangle &
        \textlangle zin\textrangle &
        \textlangle je\textrangle &
        \textlangle sayo\textrangle &
        \textlangle zeyo\textrangle \Tstrut\\
        & & & & & \Bstrut\\
    \hline
\end{tabular}

\subsubsection{Adpositions}
Sumsgiwa uses prepositions that take uninflected nouns. The prepositional phrases follow the verb. However, some functions usually covered by adpositions in other languages are covered by the \acrlong{dat} case.
%TODO: example