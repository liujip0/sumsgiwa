\documentclass[12pt, letterpaper]{article}
\usepackage[letterpaper, portrait, margin=1in]{geometry}
\usepackage{graphicx} % Required for inserting images

\usepackage[skip=1em, indent=0pt]{parskip}

\usepackage{lmodern}
\usepackage{fontspec}

\usepackage[safe,T1]{tipa}
\DeclareFontFamilySubstitution{T3}{\familydefault}{cmr}
\usepackage{textcomp}
% \DeclareFontFamily{T1}{lmr}{\hyphenchar\font\m@ne}

\newfontface{\sumsgiwafont}{font1}
    [
        Path=./,
        Extension=.ttf,
        Renderer=HarfBuzz,
        RawFeature={+kern, +liga}
    ]
\usepackage{expl3}
\ExplSyntaxOn
\newcommand{\sumsgiwasub}[1]
{
  \group_begin:
  \tl_set:Nn \l_sg_input_tl {#1}
  
  % Basic replacements (unchanged)
  \regex_replace_all:nnN { sg } { 6 } \l_sg_input_tl
  \regex_replace_all:nnN { sd } { 5 } \l_sg_input_tl
  \regex_replace_all:nnN { sb } { 4 } \l_sg_input_tl
  \regex_replace_all:nnN { \c{v}\cB\{s\cE\}g } { 3 } \l_sg_input_tl
  \regex_replace_all:nnN { \c{v}\cB\{s\cE\}d } { 2 } \l_sg_input_tl
  \regex_replace_all:nnN { \c{v}\cB\{s\cE\}b } { 1 } \l_sg_input_tl
  \regex_replace_all:nnN { \c{v}\cB\{c\cE\} } { C } \l_sg_input_tl
  \regex_replace_all:nnN { \c{.}\cB\{z\cE\} } { 9 } \l_sg_input_tl
  \regex_replace_all:nnN { \c{v}\cB\{z\cE\} } { Z } \l_sg_input_tl
  \regex_replace_all:nnN { \c{v}\cB\{s\cE\} } { S } \l_sg_input_tl
  \regex_replace_all:nnN { o } {} \l_sg_input_tl
  
  % Conditional replacements using regex lookahead
  \regex_replace_all:nnN { m([^aeiou]) } { M\1 } \l_sg_input_tl  % Before consonants
  \regex_replace_all:nnN { n([^aeiou]) } { N\1 } \l_sg_input_tl
  \regex_replace_all:nnN { ñ([^aeiou]) } { 8\1 } \l_sg_input_tl
  \regex_replace_all:nnN { ñ([aeiou]) } { 7\1 } \l_sg_input_tl   % Before vowels
  
  % Handle word-final positions (no following character)
  \regex_replace_all:nnN { m\Z } { M } \l_sg_input_tl
  \regex_replace_all:nnN { n\Z } { N } \l_sg_input_tl
  \regex_replace_all:nnN { ñ\Z } { 8 } \l_sg_input_tl
  
  \tl_use:N \l_sg_input_tl
  \group_end:
}
\ExplSyntaxOff

\DeclareRobustCommand{\SG}[1]{{%
  {\sumsgiwafont\sumsgiwasub{#1}}%
}}

\usepackage{titlesec}
\setcounter{secnumdepth}{5}
\setcounter{tocdepth}{5}
\titleformat{\paragraph}
    [hang]
    {\normalfont\small\bfseries}
    {\theparagraph}
    {10pt}
    {}
    [\noindent]
\titleformat{\subparagraph}
    [hang]
    {\normalfont\footnotesize\bfseries}
    {\thesubparagraph}
    {7pt}
    {}
    [\noindent]
\titlespacing*{\paragraph}{0pt}{3.25ex plus 1ex minus .2ex}{-20pt}
\titlespacing*{\subparagraph}{0pt}{3.25ex plus 1ex minus .2ex}{-20pt}

\usepackage{hyperref}
\hypersetup{
    pdfborderstyle={/S/U},
    linktoc=all
}
\usepackage{nameref}
\newcommand*{\fullref}[1]{\hyperref[{#1}]{\ref*{#1} \nameref*{#1}}}

\usepackage{url}

\usepackage{multicol}

\usepackage[english, provide=*]{babel}
\usepackage[autostyle, english = american]{csquotes}
\MakeOuterQuote{"}

\usepackage{datatool}[=v2.32]
\usepackage[mcolblock, leipzighyper, toc, glosses, symbols, acronyms]{leipzig}
\usepackage{glossaries}
\usepackage{glossary-mcols}
\renewcommand{\leipzigname}{Glossing Abbreviations}
\setglossarystyle{mcolindex}
\renewcommand{\leipzigfont}[1]{\textsc{#1}}
\makeglossaries
\newleipzig{Fex}{1ex}{first person exclusive}
\newleipzig{Fin}{1in}{first person inclusive}
\newleipzig{S}{2}{second person}
\newleipzig{Spol}{2pol}{second person polite}
\newleipzig{Shum}{2hum}{second person humiliative}
\newleipzig{Sfam}{2fam}{second person familiar}
\newleipzig{T}{3}{third person}
\newleipzig{an}{an}{animate}
\newleipzig{deo}{deo}{deontic}
\newleipzig{hab}{hab}{habitual}
\newleipzig{hg}{h}{human}
\newleipzig{inan}{inan}{inanimate}
\newleipzig{iter}{iter}{iterative}
\newleipzig{med}{med}{medial}
\newleipzig{opt}{opt}{optative}
\newleipzig{pos}{pos}{positive}
\newleipzig{sjv}{sjv}{subjunctive}
\newleipzig{yn}{yn}{yes/no question}

\newacronym{tam}{TAM}{tense-aspect-mood}

\newglossaryentry{agent}{
    name=agent,
    description={The argument of a transitive verb that is performing the action}
}
\newglossaryentry{patient}{
    name=patient,
    description={The argument of a transitive verb that is receiving the action}
}
\newglossaryentry{subject}{
    name=subject,
    description={The sole argument of an intransitive verb}
}
\newglossaryentry{realis}{
    name=realis,
    description={A statement of fact}
}
\newglossaryentry{irrealis}{
    name=irrealis,
    description={A certain situation or action that is not known to have happened at the moment the speaker is talking}
}

%TODO add symbols using nomenclature or glossary
% - separable morphemes in the same word
% [] IPA narrow transcription
% // IPA broad transcription
% <> romanization
\newcommand\Glossfull[2][]{\Acrlong[#1]{#2} \acrshort[#1]{#2}}
\newcommand\glossfull[2][]{\acrlong[#1]{#2} \acrshort[#1]{#2}}

\usepackage{multirow}
\renewcommand{\arraystretch}{1.1}
% Define "struts" as suggested by Claudio Beccari in
% a piece in TeX and TUG News, Vol. 2, 1993.
\newcommand\Tstrut{\rule{0pt}{2.6ex}}       % "top" strut
\newcommand\Bstrut{\rule[-0.9ex]{0pt}{0pt}} % "bottom" strut
\newcommand{\TBstrut}{\Tstrut\Bstrut}
\usepackage{array}
\usepackage{longtable}

\usepackage{enumitem}

\usepackage{hanging}

\newcommand{\entry}[4]
    {\textbf{#1}\ {/\textipa{#2}/}\ \textit{#3}\ $\bullet$\ {#4}}

\usepackage{expex}
\definelingstyle{defaultgloss}{
    everygla={},
    belowglpreambleskip=-1.3em,
    aboveglftskip=-1em
}

\newcommand{\oct}[1]{#1\textsubscript{8}}
\newcommand{\dec}[1]{#1\textsubscript{10}}
\newcommand{\doz}[1]{#1\textsubscript{12}}

\title{Sumsgiwa}
\author{liujip0}
\date{December 2024}

\begin{document}
\maketitle
\tableofcontents
\clearpage

\addcontentsline{toc}{section}{Util}
\section*{Util}

\subsection{gen - language text generator}
\url{https://www.zompist.com/gen.html}

{[}Double click for Categories{]}
\iffalse
C=šždkgszpbtčjywcżmnñ
V=iueoa
N=nñm
S=šs
P=bdg
\fi

{[}Double click for Syllable types{]}
\iffalse
CV
SPV
CVN
SPVN
\fi

\section{Introduction}

\section{Typology}

\subsection{Word Order}
The default sentence structure in Sumsgiwa is VAP/VS (VSO). Negative clauses are AVP/SV (SVO), and nouns can be topicalized by moving them to the front of the clause.

\subsection{Morphological Typology}

\subsubsection{Synthesis}
Sumsgiwa is a somewhat polysynthetic language, with many morphemes occurring in each word.

\subsubsection{Fusion}
Sumsgiwa is highly agglutinative, with most morphemes having only one meaning.

\subsubsection{Morphological Processes}
Sumsgiwa inflects its words mostly through the use of prefixes. There are also a few suffixes and stem modifications.
\section{Phonology}

\subsection{Consonants}
\begin{tabular}{|c|c|c|c|c|c|c|c|c|c|c|c|c|}
    \hline
    & \multicolumn{2}{c|}{Bilabial} &
        \multicolumn{2}{c|}{Alveolar} &
        \multicolumn{2}{c|}{Postalveolar} &
        \multicolumn{2}{c|}{Palatal} &
        \multicolumn{2}{c|}{Labiovelar} &
        \multicolumn{2}{c|}{Velar} \TBstrut\\
    \hline

    \multirow{3}{*}{Plosive} &
        \textipa{[p\super h]} & \textipa{[b]} &
        \textipa{[t\super h]} & \textipa{[d]} &
        \multicolumn{2}{c|}{} &
        \multicolumn{2}{c|}{} &
        \multicolumn{2}{c|}{} &
        \textipa{[k\super h]} & \textipa{[g]} \Tstrut\\

        & \textipa{/p/} & \textipa{/b/} &
        \textipa{/t/} & \textipa{/d/} &
        \multicolumn{2}{c|}{} &
        \multicolumn{2}{c|}{} &
        \multicolumn{2}{c|}{} &
        \textipa{/k/} & \textipa{/g/} \\

        & \textlangle p\textrangle &
        \textlangle b\textrangle &
        \textlangle t\textrangle &
        \textlangle d\textrangle &
        \multicolumn{2}{c|}{} &
        \multicolumn{2}{c|}{} &
        \multicolumn{2}{c|}{} &
        \textlangle k\textrangle &
        \textlangle g\textrangle \Bstrut\\
    \hline

    \multirow{3}{*}{Nasal} &
        & \textipa{[m]} &
        & \textipa{[n]} &
        \multicolumn{2}{c|}{} &
        \multicolumn{2}{c|}{} &
        \multicolumn{2}{c|}{} &
        & \textipa{[N]} \Tstrut\\
        
        & & \textipa{/m/} &
        & \textipa{/n/} &
        \multicolumn{2}{c|}{} &
        \multicolumn{2}{c|}{} &
        \multicolumn{2}{c|}{} &
        & \textipa{/N/} \\
        
        & & \textlangle m\textrangle &
        & \textlangle n\textrangle &
        \multicolumn{2}{c|}{} &
        \multicolumn{2}{c|}{} &
        \multicolumn{2}{c|}{} &
        & \textlangle ñ\textrangle \Bstrut\\
    \hline
    
    \multirow{3}{*}{Fricative} &
        \multicolumn{2}{c|}{} &
        \textipa{[s]} & \textipa{[z]} &
        \textipa{[S]} & \textipa{[Z]} &
        \multicolumn{2}{c|}{} &
        \multicolumn{2}{c|}{} &
        \multicolumn{2}{c|}{} \Tstrut\\
        
        & \multicolumn{2}{c|}{} &
        \textipa{/s/} & \textipa{/z/} &
        \textipa{/S/} & \textipa{/Z/} &
        \multicolumn{2}{c|}{} &
        \multicolumn{2}{c|}{} &
        \multicolumn{2}{c|}{} \\
        
        & \multicolumn{2}{c|}{} &
        \textlangle s\textrangle &
        \textlangle z\textrangle &
        \textlangle š\textrangle &
        \textlangle ž\textrangle &
        \multicolumn{2}{c|}{} &
        \multicolumn{2}{c|}{} &
        \multicolumn{2}{c|}{} \Bstrut\\
    \hline

    \multirow{3}{*}{Approximant} &
        \multicolumn{2}{c|}{} &
        \multicolumn{2}{c|}{} &
        \multicolumn{2}{c|}{} &
        \multicolumn{2}{c|}{\textipa{[j]}} &
        \multicolumn{2}{c|}{\textipa{[w]}} &
        \multicolumn{2}{c|}{} \Tstrut\\
        
        & \multicolumn{2}{c|}{} &
        \multicolumn{2}{c|}{} &
        \multicolumn{2}{c|}{} &
        \multicolumn{2}{c|}{\textipa{/j/}} &
        \multicolumn{2}{c|}{\textipa{/w/}} &
        \multicolumn{2}{c|}{} \\
        
        & \multicolumn{2}{c|}{} &
        \multicolumn{2}{c|}{} &
        \multicolumn{2}{c|}{} &
        \multicolumn{2}{c|}{\textlangle y\textrangle} &
        \multicolumn{2}{c|}{\textlangle w\textrangle} &
        \multicolumn{2}{c|}{} \Bstrut\\
    \hline

    \multirow{3}{*}{Affricate} &
        \multicolumn{2}{c|}{} &
        \textipa{[\t{ts}]} & \textipa{[\t{dz}]} &
        \textipa{[\t{tS}]} & \textipa{[\t{dZ}]} &
        \multicolumn{2}{c|}{} &
        \multicolumn{2}{c|}{} &
        \multicolumn{2}{c|}{} \Tstrut\\
        
        & \multicolumn{2}{c|}{} &
        \textipa{/\t{ts}/} & \textipa{/\t{dz}/} &
        \textipa{/\t{tS}/} & \textipa{/\t{dZ}/} &
        \multicolumn{2}{c|}{} &
        \multicolumn{2}{c|}{} &
        \multicolumn{2}{c|}{} \\
        
        & \multicolumn{2}{c|}{} &
        \textlangle c\textrangle &
        \textlangle ż\textrangle &
        \textlangle č\textrangle &
        \textlangle j\textrangle &
        \multicolumn{2}{c|}{} &
        \multicolumn{2}{c|}{} &
        \multicolumn{2}{c|}{} \Bstrut\\
    \hline
\end{tabular}

\subsection{Vowels}
\begin{tabular}{|c|c|c|}
    \hline
    & Front & Back \TBstrut\\
    \hline

    \multirow{3}{3em}{Close} &
        \textipa{[i]} & \textipa{[u]} \Tstrut\\
        & \textipa{/i/} & \textipa{/u/} \\
        & \textlangle i\textrangle & \textlangle u\textrangle \Bstrut\\
    \hline

    \multirow{3}{3em}{Mid} &
        \textipa{[e $\sim$ E]} & \textipa{[o]} \Tstrut\\
        & \textipa{/e/} & \textipa{/o/} \\
        & \textlangle e\textrangle & \textlangle o\textrangle \Bstrut\\
    \hline

    \multirow{3}{3em}{Open} &
        \textipa{[a]} & \Tstrut\\
        & \textipa{/a/} & \\    
        & \textlangle a\textrangle & \Bstrut\\
    \hline
\end{tabular}

\subsection{Syllable Structure}
Syllables in Sumsgiwa consist of an onset, a nucleus, and a rhyme. Most syllables are open. Syllable possibilities:

\noindent\\
CV(N)\\
C = p, b, t, d, k, g, m, n, ñ, s, z, š, ž, y, w, c, ż, č, j\\
V = i, u, e, o, a\\
N = m, n, ñ

\noindent\\
SPV(N)\\
S = s, š\\
P = b, d, g\\
V = i, u, e, o, a\\
N = m, n, ñ

\subsection{Stress}
Stress always falls on the penultimate syllable of the word stem and does not move when prefixes or suffixes are added.
\section{Grammar}

\subsection{Parts of Speech}

{
  \newcommand{\TableRowPrefix}[2]{
    \Glossfull{#1} &
    \textlangle #2\textrangle - & \SG{#2} \TBstrut\\
    \hline
  }
  \newcommand{\TableRowSuffix}[2]{
    \Glossfull{#1} &
    -\textlangle #2\textrangle & \SG{#2} \TBstrut\\
    \hline
  }
  \newcommand{\TableRowInfix}[2]{
    \Glossfull{#1} &
    -\textlangle #2\textrangle - & \SG{#2} \TBstrut\\
    \hline
  }

  \subsubsection{Nouns}

  \paragraph{Noun Phrases}
  Noun phrases are structured as follows:

  {[}Adposition{]}\\
  {[}Ordinal{]}\\
  Case-NOUN STEM-{[}Augmentative/Diminutive{]}\\
  {[}Adjective(s){]}\\
  {[}Comparative{}]\\
  {[}Cardinal/Quantifier{]}\\
  {[}Demonstrative{]}\\
  {[}Genitive{]}\\
  {[}Relative Phrase{]}

  {[]} = optional part of noun phrase
  %TODO: example

  \paragraph{Noun Cases}
  There are 5 cases in Sumsgiwa indicated using prefixes:
  \begin{description}
    \item[\Glossfull{nom}] \Gls{subject} of an intransitive verb
    \item[\Glossfull{erg}] \Gls{agent} of a transitive verb
    \item[\Glossfull{acc}] \Gls{patient} of a transitive verb
    \item[\Glossfull{dat}] Indirect object of a verb
    \item[\Glossfull{gen}] Modifier of another noun
  \end{description}
  \begin{tabular}{|l|c|c|}
    \hline
    \TableRowPrefix{nom}{ja}
    \TableRowPrefix{erg}{di}
    \TableRowPrefix{acc}{\v{s}a}
    \TableRowPrefix{dat}{\v{s}o}
    \TableRowPrefix{gen}{ne}
  \end{tabular}

  \paragraph{Gender}
  There are 3 genders, and nouns are categorized purely through semantics, rather than through phonology or spelling.
  \begin{description}
    \item[\Glossfull{hg}] Humans, body parts, nouns relating to language
    \item[\Glossfull{an}] Animals, thoughts
    \item[\Glossfull{inan}] Objects, plants, fungi
  \end{description}
  Both adjectives and verbs must agree with nouns in gender.

  \subsubsection{Verbs}

  \paragraph{Verb Phrases}
  Verb phrases are structured as follows:

  {[}Question Particle{]}\\
  Mood-Aspect-VERB STEM-Gender of S-Gender of A-Gender of P\\
  {[}Adverb(s){]}

  {[]} = optional part of verb phrase
  %TODO: example

  \paragraph{Mood}
  Verbs have 4 moods marked using prefixes:
  \begin{description}
    \item[\Glossfull{ind}] Default, all \gls{realis} statements
    \item[\Glossfull{opt}] Wishes, hopes, and desires
    \item[\Glossfull{deo}] How things "ought" to be
    \item[\Glossfull{sjv}] All other \gls{irrealis} statements
  \end{description}
  \begin{tabular}{|l|c|c|}
    \hline
    \Glossfull{ind} &
    Ø- or \textlangle s\textrangle - & \SG{s} \TBstrut\\
    \hline
    \Glossfull{opt} &
    \textlangle \v{s}d\textrangle - & \SG{\v{s}d} \TBstrut\\
    \hline
    \Glossfull{deo} &
    \textlangle \v{s}u\v{c}\textrangle - & \SG{\v{s}u\v{c}} \TBstrut\\
    \hline
    \Glossfull{sjv} &
    \textlangle y\textrangle - & \SG{y} \TBstrut\\
    \hline
  \end{tabular}

  Imperative statements use the \acrlong{deo} mood. [See \fullref{04_03_03_Imperatives}]

  \paragraph{Aspect}
  Verbs have 4 aspects marked with prefixes:
  \begin{description}
    \item[\Glossfull{pfv}] Complete action as a single event in time
    \item[\Glossfull{hab}] Habitual actions (repetition over multiple occasions)
    \item[\Glossfull{iter}] Repeated actions (repetition at a single occasion)
    \item[\Glossfull{prog}] Action in progress at a specific time (incl. continuous)
  \end{description}
  \begin{tabular}{|l|c|c|}
    \hline
    \Glossfull{pfv} &
    -Ø- or -\textlangle e\textrangle - & \SG{se} \TBstrut\\
    \hline
    \Glossfull{hab} &
    -\textlangle ini\textrangle - & \SG{sini} \TBstrut\\
    \hline
    \Glossfull{iter} &
    -\textlangle i\textrangle - & \SG{si} \TBstrut\\
    \hline
    \Glossfull{prog} &
    -\textlangle a\textrangle - & \SG{sa} \TBstrut\\
    \hline
  \end{tabular}

  \paragraph{Agreement with Nouns} \label{04_01_02_04_Agreement with Nouns}
  Verbs agree with their subject, agent, and patient arguments in gender.

  \subparagraph{Subject Gender}
  \begin{tabular}{|l|c|c|}
    \hline
    \TableRowSuffix{hg}{po}
    \TableRowSuffix{an}{\v{z}u}
    \TableRowSuffix{inan}{\v{s}bi}
  \end{tabular}

  \subparagraph{Agent Gender}
  \begin{tabular}{|l|c|c|}
    \hline
    \TableRowInfix{hg}{jo}
    \TableRowInfix{an}{\v{c}e}
    \TableRowInfix{an}{sba}
  \end{tabular}

  \subparagraph{Patient Gender}
  \begin{tabular}{|l|c|c|}
    \hline
    \TableRowSuffix{hg}{pon}
    \TableRowSuffix{an}{\v{s}um}
    \TableRowSuffix{inan}{\v{z}o}
  \end{tabular}

  \subsubsection{Adjectives}
  Adjectives are structured as follows:

  Case-ADJECTIVE STEM-Gender
  %TODO: example

  \paragraph{Agreement with Nouns}
  Adjectives agree with the nouns they modify in case and gender.

  \subparagraph{Case}
  \begin{tabular}{|l|c|c|}
    \hline
    \TableRowPrefix{nom}{\v{c}a}
    \TableRowPrefix{erg}{ti}
    \TableRowPrefix{acc}{\v{s}a}
    \TableRowPrefix{dat}{\v{s}o}
    \TableRowPrefix{gen}{ne}
  \end{tabular}

  \subparagraph{Gender}
  \begin{tabular}{|l|c|c|}
    \hline
    \TableRowSuffix{hg}{po}
    \TableRowSuffix{an}{\v{z}u}
    \TableRowSuffix{inan}{\v{s}bi}
  \end{tabular}

  \subsubsection{Adverbs}
  Adverb structure:

  Aspect-ADVERB STEM

  \textit{or}

  ADVERB STEM-Gender
  %TODO: example

  \paragraph{Agreement with Verbs}
  Adverbs follow the verbs they modify and agree with them in aspect.

  \subparagraph{Aspect}
  \begin{tabular}{|l|c|c|}
    \hline
    \Glossfull{pfv} &
    Ø- & \TBstrut\\
    \hline
    \TableRowPrefix{hab}{ni}
    \TableRowPrefix{iter}{neye}
    \TableRowPrefix{prog}{na}
  \end{tabular}

  \paragraph{Agreement with Adjectives}
  Adverbs follow the adjectives they modify and agree with them in gender.

  \subparagraph{Gender}
  \begin{tabular}{|l|c|c|}
    \hline
    \TableRowSuffix{hg}{po}
    \TableRowSuffix{an}{\v{s}u}
    \TableRowSuffix{inan}{\v{s}be}
  \end{tabular}
}

\subsubsection{Pronouns}
There are different pronoun forms for each of the 5 cases. Pronouns in Sumsgiwa distinguish between 1st person exclusive and 1st person inclusive, and there are 3 politeness levels for 2nd person pronouns. In addition, there are 3 sets of pronouns for each of the 3 genders.
\begin{description}
  \item[\Glossfull{Fex}] 1st person, excluding the hearer
  \item[\Glossfull{Fin}] 1st person, including the hearer
  \item[\Glossfull{Spol}] 2nd person, used in formal or neutral circumstances
  \item[\Glossfull{Shum}] 2nd person, used when speaking to figures of authority, elders, etc.
  \item[\Glossfull{Sfam}] 2nd person, used when speaking to friends and/or family members
  \item[\Glossfull{T}] 3rd person
\end{description}

{
  \newcommand{\TableRow}[6]{
    \multirow{2}{5em}{\Glossfull{#1}} &
    \textlangle #2\textrangle &
    \textlangle #3\textrangle &
    \textlangle #4\textrangle &
    \textlangle #5\textrangle &
    \textlangle #6\textrangle \Tstrut\\
    & \SG{#2} & \SG{#3} & \SG{#4} & \SG{#5} & \SG{#6} \Bstrut\\
    \hline
  }
  \newcommand{\TableRowTall}[6]{
    \multirow{4}{5em}{\Glossfull{#1}} &
    \textlangle #2\textrangle &
    \textlangle #3\textrangle &
    \textlangle #4\textrangle &
    \textlangle #5\textrangle &
    \textlangle #6\textrangle \Tstrut\\
    & \SG{#2} & \SG{#3} & \SG{#4} & \SG{#5} & \SG{#6} \\
    & & & & & \\
    & & & & & \Bstrut\\
    \hline
  }
  \newcommand{\TableHead}{\hline
    & \Glossfull{nom} &
    \Glossfull{erg} &
    \Glossfull{acc} &
    \Glossfull{dat} &
    \Glossfull{gen} \TBstrut\\
  \hline}

  \paragraph{Human}

  \begin{tabular}{|m{5em}|m{5em}|m{4em}|m{5em}|m{4em}|m{4em}|}
    \TableHead

    \TableRowTall{Fex}{yem}{ya}{yu}{ye\.{z}o}{yañi}

    \TableRowTall{Fin}{\v{z}em}{\v{z}a}{\v{z}u}{\v{z}e\.{z}o}{\v{z}iñi}

    \TableRowTall{Spol}
    {ji\v{s}bem}{ji\v{s}ba}{ji\v{s}bu}{ji\.{z}o}{ji\v{s}i}

    \TableRowTall{Shum}
    {\.{z}e\v{s}bem}{\.{z}e\v{s}ba}{\.{z}e\v{s}bu}{\.{z}e\v{z}o}{\.{z}e\v{s}i}

    \TableRowTall{Sfam}{\v{s}bem}{\v{s}ba}{\v{s}bu}{\v{s}be\.{z}o}{\v{s}biñi}

    \TableRow{T}{\v{c}em}{\v{c}a}{\v{c}u}{\v{c}u\.{z}o}{\v{c}añi}
  \end{tabular}

  \paragraph{Animate}

  \begin{tabular}{|m{5em}|m{5em}|m{4em}|m{5em}|m{4em}|m{4em}|}
    \TableHead

    \TableRow{S}{consi}{consan}{conso}{sonsu\v{c}e}{consu\v{c}i}

    \TableRow{T}{si}{san}{so}{su\v{c}e}{su\v{c}i}
  \end{tabular}

  \paragraph{Inanimate}
  \begin{tabular}{|m{5em}|m{5em}|m{4em}|m{5em}|m{4em}|m{4em}|}
    \TableHead

    \TableRow{T}{yo}{zin}{je}{sayo}{zeyo}
  \end{tabular}
}

\subsubsection{Adpositions}
Sumsgiwa uses prepositions that take uninflected nouns. The prepositional phrases follow the verb. However, some functions usually covered by adpositions in other languages are covered by the \acrlong{dat} case.
%TODO: example

\subsection{Noun and Noun-Phrase Operations}

\subsubsection{Noun-Noun Compounding}
Nouns are compounded by juxtaposition, with the new combined form inflected as one word.
%TODO: example

\subsubsection{Denominalization}
There are two ways nouns can be verbalized, with different meanings. In both cases, morphological changes are applied to the uninflected forms of the nouns.

The prefix \textlangle do\textrangle - converts the noun into a verb meaning "to become [noun]."

\ex~[glstyle=nlevel]
\begingl
\glpreamble \SGwithRom{do\v{c}uzipo \v{c}em}
\endpreamble
Ø-[{\Ind}-]@
Ø-[{\Pfv}-]@
do-[{\Vbz}-]@
\v{c}uzi[man]@
-po[-{\Hg}]
\v{c}em[{\T}.{\Hg}.{\Nom}]
\glft "He became a man." (e.g. "He transitioned.")
\endgl
\xe

The prefix \textlangle ke\textrangle - changes the noun into a verb that means "to have [noun]."

\ex~[glstyle=nlevel]
\begingl
\glpreamble \SGwithRom{sakebi\v{s}epo \v{c}em}
\endpreamble
s-[{\Ind}-]@
a-[{\Prog}-]@
ke-[{\Vbz}-]@
bi\v{s}e[table]@
-po[-{\Hg}]
\v{c}em[{\T}.{\Hg}.{\Nom}]
\glft "They own tables."
\endgl
\xe

\subsubsection{Adjectivization}
A noun can take adjectival case and gender markings to become an adjective meaning "similar to [noun]."

\ex~[glstyle=nlevel]
\begingl
\glpreamble \SGwithRom{jakengo \.{z}e\v{s}i \v{c}adezu\v{z}u}
\endpreamble
ja-[{\Nom}-]@
kengo[action]
\.{z}e\v{s}i[{\Shum}.{\Hg}.{\Gen}]
\v{c}a-[{\Nom}.{\Adjz}-]@
dezu[jaguar]@
-\v{z}u[-{\An}.{\Adjz}]
\glft "Your actions are jaguar-like!"
\endgl
\xe

\subsubsection{Determiners}

\paragraph{Demonstratives}
Demonstratives agree with the nouns they modify in gender, and are split into 3 categories. They go after the nouns they modify.
\begin{description}
  \item[\Glossfull{prox}] Close to the speaker
  \item[\Glossfull{med}] Close to the listener
  \item[\Glossfull{dist}] Far from both the speaker and the listener
\end{description}
\begin{tabular}{|l|c|c|c|}
  \hline
  & \Glossfull{prox} &
  \Glossfull{med} &
  \Glossfull{dist} \TBstrut\\
  \hline

  \multirow{2}{*}{\Glossfull{hg}} &
  \textlangle ke\v{s}i\textrangle &
  \textlangle zi\v{s}go\textrangle &
  \textlangle simsbe\textrangle \Tstrut\\
  & & & \Bstrut\\
  \hline

  \multirow{2}{*}{\Glossfull{an}} &
  \textlangle gunzin\textrangle &
  \textlangle sgaku\textrangle &
  \textlangle densi\textrangle \Tstrut\\
  & & & \Bstrut\\
  \hline

  \multirow{2}{*}{\Glossfull{inan}} &
  \textlangle dazo\textrangle &
  \textlangle yezi\textrangle &
  \textlangle \v{s}osu\textrangle \Tstrut\\
  & & & \Bstrut\\
  \hline
\end{tabular}

Sumsgiwa does not have demonstrative pronouns, so it uses a demonstrative adjective and a generic noun to specify the same meaning.

\subsection{Verb and Verb-Phrase Operations}

\subsubsection{Negation}
When verbs are negated, the first consonant of the verb stem changes from a voiced consonant to an unvoiced one, or vice versa, and the clause-level word order changes from VAP/VS (VSO) to AVP/SV (SVO). If the initial consonant is \textlangle y\textrangle~or \textlangle w\textrangle, it becomes \textlangle n\textrangle, \textlangle m\textrangle, or \textlangle ñ\textrangle, and vice versa. Which one the consonant becomes is unpredictable and must be memorized on a per-word basis.
%TODO: example

\subsubsection{Interrogatives}

\paragraph{Yes/No Questions}
Yes/no questions are marked by placing the question particle \textlangle \v{s}gezu\textrangle~at the beginning of the sentence.
%TODO: example

\paragraph{Content Questions}
Content questions use the same interrogative marker as yes/no questions, \textlangle \v{s}gezu\textrangle, but also use the following question words in situ:

{
  \newcommand{\TableRow}[2]{
    #1 & \textlangle #2\textrangle & \SG{#2} \TBstrut\\
    \hline
  }

  \begin{tabular}{|c|c|c|}
    \hline
    \TableRow{\Acrlong{nom} Noun}{sensu}
    \TableRow{\Acrlong{erg} Noun}{senyu}
    \TableRow{\Acrlong{acc} Noun}{senge}
    \TableRow{\Acrlong{dat} Noun}{senbi}
    \TableRow{\Acrlong{gen} Noun}{sen\v{z}o}
    \TableRow{Time Adverb}{\v{s}anbi}
    \TableRow{Location Adverb}{\v{s}anzim}
    \TableRow{Manner Adverb}{\v{s}an\v{z}e}
    \TableRow{Purpose/Reason Adverb}{\v{s}ansum}
  \end{tabular}
}
%TODO: example

\subsubsection{Imperatives} \label{04_03_03_Imperatives}
Imperatives are in the \acrlong{deo} mood and have either the imperative particle \textlangle sesum\textrangle~or the negative imperative particle \textlangle dosika\textrangle~in front of the sentence. When the speaker is telling the hearer to do something, the agent is dropped and the patient goes into the nominative case.
%TODO: example

When both the agent and patient are present in the sentence, the speaker is telling the hearer to ask the agent to perform the action.
%TODO: example

\paragraph{Polite Imperatives}
Polite imperatives use the \acrlong{opt} mood instead of the \acrlong{deo} mood. Additionally, the agent is explicitly expressed using either the \glossfull{Spol} or \glossfull{Shum} pronouns.

\subsubsection{Causatives}
Causatives in Sumsgiwa are marked with the prefix \textlangle kaga\textrangle- on the fully-inflected verb. The tense/aspect inflections are still for the action itself rather than for causing the action. The causer takes the ergative case while the causee will take either the accusative or dative cases. The dative case is when the causee has little agency in the situation, while the accusative case is when they have some amount of agency.
%TODO: example

\subsubsection{Possessor Raising}
Intransitive verbs with possessed subjects can be expressed as transitive verbs with the possessee as the agent and the possessor as the patient.
%TODO: example

\subsubsection{Argument Omission}
Any argument of a verb can be omitted as long as the omitted noun is clear from context.
%TODO: example

\subsubsection{Verb Compounding}

\paragraph{Noun Incorporation}

\subparagraph{Subject/Agent Incorporation}
Subjects and agents are incorporated by moving their uninflected forms before the verb stem. Any prefixes that go on the verb will move to before the incorporated noun.
%TODO: example

\subparagraph{Patient Incorporation}
Patients are incorporated by moving their uninflected forms after the verb stem. Any suffixes the verb may take will go after the incorporated patient.
%TODO: example

\paragraph{Verb-Verb Incorporation}
Verbs are compounded by juxtaposition, with inflections applied to the new combined word.
%TODO: example

\subsubsection{Nominalization}

\paragraph{Action Nominalization}
Action nominalization is achieved by adding case markings onto the uninflected form of the verb.
%TODO: example

\paragraph{Agent/Subject Nominalization}
Sumsgiwa nominalizes the typical agent or subject of a verb by adding the prefix \textlangle juzi\textrangle- to its uninflected form.
%TODO: example

\paragraph{Patient Nominalization}
The typical patients of verbs are nominalized by adding the suffix -\textlangle sokuñ\textrangle~to the verbs' uninflected forms.
%TODO: example

\paragraph{Instrument Nominalization}
Sumsgiwa nominalizes the typical instruments of a verb by adding the suffix -\textlangle sodise\textrangle~to its uninflected form.
%TODO: example

\paragraph{Location Nominalization}
Sumsgiwa nominalizes the typical locations where a verb happens by adding the suffix -\textlangle gomsa\textrangle~to its uninflected form.
%TODO: example

\paragraph{Product Nominalization}
Sumsgiwa nominalizes the typical products of the action represented by a verb by adding the suffix -\textlangle sigo\textrangle~to its uninflected form.
%TODO: example

\subsection{Clause Operations}

\subsubsection{Left-Dislocation}
Any argument of a verb can be left-dislocated by adding the suffix -\textlangle sugu\textrangle.
%TODO: example

\subsubsection{Cleft Constructions}
Sumsgiwa has cleft constructions consisting of the noun phrase, the predicate nominal copula [See \fullref{04_05_02_Predicate Nominals}], and a headless relative clause [See \fullref{04_06_03_Relative Clauses}], in that order.
%TODO: example
\subsection{Simple Sentences}

\subsubsection{Simple Declarative Sentences}
Simple declarative sentences in Sumsgiwa are expressed with VAP/VS word order. Transitive verbs agree with both the agent and patient, and intransitive verbs agree with the subject [See \fullref{04_01_02_04_Agreement with Nouns}].
%TODO: example

\subsubsection{Predicate Nominals} \label{04_05_02_Predicate Nominals}
Predicate nominals use a defective \glossfull{cop} verb, \textlangle se\textrangle, that inflects for mood but not aspect or gender. These constructions encompass proper inclusion, where something is asserted to be among the class specified by the nominal predicate, and equative clauses, where two things are asserted to be the same.
%TODO: example

\subsubsection{Predicate Adjectives}
Predicate adjectives in Sumsgiwa consist of the noun and properly inflected adjective juxtaposed with each other, with no copular verbs or particles.
%TODO: example

\subsubsection{Predicate Locatives}
Predicate locative constructions use the locative adposition, \textlangle \v{s}im\textrangle, as a defective verb, inflecting it for mood but not for aspect or gender.
%TODO: example

\subsubsection{Existentials}
Existentials use the same copular verb as predicate nominals, \textlangle se\textrangle, and also only inflect it for mood.
%TODO: example

\subsubsection{Possessive Clauses}
Possessive clauses use the same verb as predicate locatives, \textlangle \v{s}im\textrangle, also only inflected for mood. The possessor is placed in the genitive case.
%TODO: example
\subsection{Clause Combinations}

\subsubsection{Complement Clauses}
Sumsgiwa forms complement clauses by adding a case-marking prefix onto the verb.
%TODO: example

\subsubsection{Adverbial Clauses}
Adverbial clauses use an adposition combined with a relative clause. The adposition, word expressing the function of the adverbial clause, and relativizer are shortened into one word.

%TODO: "At the time that…" shortened into one word
%TODO: example

\subsubsection{Relative Clauses} \label{04_06_03_Relative Clauses}
Relative clauses come after the head noun. They are introduced with the relativizer \textlangle \v{s}u\v{z}on\textrangle~and use a pronoun retention strategy for case recoverability. All clause elements can be relativized.
%TODO: example

\subsubsection{Coordination}

\paragraph{Conjunction}
There is no special morphosyntax for conjunction, with clauses simply juxtaposed next to each other. So, any two clauses in Sumsgiwa can be said to be conjoined.
%TODO: example

\paragraph{Disjunction}
Disjunction is expressed with the particle \textlangle zi\v{s}e\textrangle~inserted between the two clauses.
%TODO: example
\section{Quantifiers}

\subsection{Numerals}
There are 3 numeral systems in Sumsgiwa, each modifying a different gender of noun. The \acrlong{inan} numerals are used for counting.

\subsubsection{Human}
Numerals modifying \acrlong{hg} nouns are base 8.

\begin{longtable}[l]{|c|c|c|c|}
    \hline
    \textbf{Octal} &
        \textbf{Decimal} &
        \textbf{Sumsgiwa} & \TBstrut\\
    \hline
    \endhead

    \oct{0} & \dec{0} &
        \textlangle pun\v{z}on\textrangle & \TBstrut\\
    \hline
    \oct{1} & \dec{1} &
        \textlangle \v{z}añe\textrangle & \TBstrut\\
    \hline
    \oct{2} & \dec{2} &
        \textlangle \v{s}ite\textrangle & \TBstrut\\
    \hline
    \oct{3} & \dec{3} &
        \textlangle \v{z}oye\textrangle & \TBstrut\\
    \hline
    \oct{4} & \dec{4} &
        \textlangle da\v{z}i\textrangle & \TBstrut\\
    \hline
    \oct{5} & \dec{5} &
        \textlangle kindu\textrangle & \TBstrut\\
    \hline
    \oct{6} & \dec{6} &
        \textlangle \v{s}oku\textrangle & \TBstrut\\
    \hline
    \oct{7} & \dec{7} &
        \textlangle pa\v{z}u\textrangle & \TBstrut\\
    \hline
    \multirow{2}{*}{\oct{10}} & \multirow{2}{*}{\dec{8}} &
        \textlangle yesa\textrangle & \Tstrut\\
        & & \textlangle \v{z}aye\v{s}a\textrangle & \Bstrut\\
    \hline
    
    \multirow{2}{*}{\oct{11}} & \multirow{2}{*}{\dec{9}} &
        \textlangle yesa \v{z}añe\textrangle & \Tstrut\\
        & & \textlangle \v{z}aye\v{s}a \v{z}añe\textrangle & \Bstrut\\
    \hline
    \multirow{2}{*}{\oct{12}} & \multirow{2}{*}{\dec{10}} &
        \textlangle yesa \v{s}ite\textrangle & \Tstrut\\
        & & \textlangle \v{z}aye\v{s}a \v{s}ite\textrangle & \Bstrut\\
    \hline
    \multicolumn{4}{|c|}{\dots} \TBstrut\\
    \hline
    
    \oct{20} & \dec{16} &
        \textlangle \v{s}iye\v{s}a\textrangle & \TBstrut\\
    \hline
    \oct{21} & \dec{17} &
        \textlangle \v{s}iye\v{s}a \v{z}añe\textrangle & \TBstrut\\
    \hline
    \multicolumn{4}{|c|}{\dots} \TBstrut\\
    \hline
    
    \oct{30} & \dec{24} &
        \textlangle \v{z}oye\v{s}a\textrangle & \TBstrut\\
    \hline
    \oct{40} & \dec{32} &
        \textlangle dayesa\textrangle & \TBstrut\\
    \hline
    \oct{50} & \dec{40} &
        \textlangle kinyesa\textrangle & \TBstrut\\
    \hline
    \oct{60} & \dec{48} &
        \textlangle \v{s}oye\v{s}a\textrangle & \TBstrut\\
    \hline
    \oct{70} & \dec{56} &
        \textlangle payesa\textrangle & \TBstrut\\
    \hline
    \multicolumn{4}{|c|}{\dots} \TBstrut\\
    \hline
    
    \multirow{2}{*}{\oct{100}} & \multirow{2}{*}{\dec{64}} &
        \textlangle \v{z}emdo\textrangle & \Tstrut\\
        & & \textlangle \v{z}azemdo\textrangle & \Bstrut\\
    \hline
    \multirow{2}{*}{\oct{101}} & \multirow{2}{*}{\dec{65}} &
        \textlangle \v{z}emdo puyesa \v{z}añe\textrangle & \Tstrut\\
        & & \textlangle \v{z}azemdo puyesa \v{z}añe\textrangle & \Bstrut\\
    \hline
    \multirow{2}{*}{\oct{102}} & \multirow{2}{*}{\dec{66}} &
        \textlangle \v{z}emdo puyesa \v{s}ite\textrangle & \Tstrut\\
        & & \textlangle \v{z}azemdo puyesa \v{s}ite\textrangle & \Bstrut\\
    \hline
    \multicolumn{4}{|c|}{\dots} \TBstrut\\
    \hline

    \multirow{2}{*}{\oct{110}} & \multirow{2}{*}{\dec{72}} &
        \textlangle \v{z}emdo yesa\textrangle & \Tstrut\\
        & & \textlangle \v{z}azemdo \v{z}aye\v{s}a\textrangle & \Bstrut\\
    \hline
    \multirow{2}{*}{\oct{120}} & \multirow{2}{*}{\dec{80}} &
        \textlangle \v{z}emdo \v{s}iye\v{s}a\textrangle & \Tstrut\\
        & & \textlangle \v{z}azemdo \v{s}iye\v{s}a\textrangle & \Bstrut\\
    \hline
    \multicolumn{4}{|c|}{\dots} \TBstrut\\
    \hline
    
    \oct{200} & \dec{128} &
        \textlangle \v{s}izemdo\textrangle & \TBstrut\\
    \hline
    \oct{300} & \dec{192} &
        \textlangle zozemdo\textrangle & \TBstrut\\
    \hline
    \oct{400} & \dec{256} &
        \textlangle da\v{z}emdo\textrangle & \TBstrut\\
    \hline
    \oct{500} & \dec{320} &
        \textlangle ki\v{z}emdo\textrangle & \TBstrut\\
    \hline
    \oct{600} & \dec{384} &
        \textlangle sozemdo\textrangle & \TBstrut\\
    \hline
    \oct{700} & \dec{448} &
        \textlangle pa\v{z}emdo\textrangle & \TBstrut\\
    \hline
\end{longtable}

\subsubsection{Animate}
Numerals modifying \acrlong{an} nouns are base 10.

\begin{longtable}[l]{|c|c|c|}
    \hline
    \textbf{Decimal} &
        \textbf{Sumsgiwa} & \TBstrut\\
    \hline
    \endhead

    \dec{0} &
        \textlangle \v{z}an\v{z}u\textrangle & \TBstrut\\
    \hline
    \dec{1} &
        \textlangle gasdi\textrangle & \TBstrut\\
    \hline
    \dec{2} &
        \textlangle dosde\textrangle & \TBstrut\\
    \hline
    \dec{3} &
        \textlangle \v{s}ikim\textrangle & \TBstrut\\
    \hline
    \dec{4} &
        \textlangle beku\textrangle & \TBstrut\\
    \hline
    \dec{5} &
        \textlangle cinsi\textrangle & \TBstrut\\
    \hline
    \dec{6} &
        \textlangle \v{s}asdum\textrangle & \TBstrut\\
    \hline
    \dec{7} &
        \textlangle ki\v{s}ge\textrangle & \TBstrut\\
    \hline
    \dec{8} &
        \textlangle \v{z}iso\textrangle & \TBstrut\\
    \hline
    \dec{9} &
        \textlangle zenbe\textrangle & \TBstrut\\
    \hline
    \multirow{2}{*}{\dec{10}} &
        \textlangle kumsgi\textrangle & \Tstrut\\
        & \textlangle gakumsgi\textrangle & \Bstrut\\
    \hline
    \multirow{2}{*}{\dec{11}} &
        \textlangle kumsgi gasdi\textrangle & \Tstrut\\
        & \textlangle gakumsgi gasdi\textrangle & \Bstrut\\
    \hline
    \multirow{2}{*}{\dec{12}} &
        \textlangle kumsgi dosde\textrangle & \Tstrut\\
        & \textlangle gakumsgi dosde\textrangle & \Bstrut\\
    \hline
    \multicolumn{3}{|c|}{\dots} \TBstrut\\
    \hline
    
    \dec{20} &
        \textlangle dokumsgi\textrangle & \TBstrut\\
    \hline
    \dec{21} &
        \textlangle dokumsgi gasdi\textrangle & \TBstrut\\
    \hline
    \multicolumn{3}{|c|}{\dots} \TBstrut\\
    \hline
    
    \dec{30} &
        \textlangle \v{s}ikumsgi\textrangle & \TBstrut\\
    \hline
    \dec{40} &
        \textlangle bekumsgi\textrangle & \TBstrut\\
    \hline
    \dec{50} &
        \textlangle cikumsgi\textrangle & \TBstrut\\
    \hline
    \dec{60} &
        \textlangle \v{s}asgumsi\textrangle & \TBstrut\\
    \hline
    \dec{70} &
        \textlangle ki\v{s}gumsi\textrangle & \TBstrut\\
    \hline
    \dec{80} &
        \textlangle \v{z}ikumsgi\textrangle & \TBstrut\\
    \hline
    \dec{90} &
        \textlangle zekumsgi\textrangle & \TBstrut\\
    \hline
    \multicolumn{3}{|c|}{\dots} \TBstrut\\
    \hline

    \multirow{2}{*}{\dec{100}} &
        \textlangle du\v{z}o\textrangle & \Tstrut\\
        & \textlangle gadu\v{z}o\textrangle & \Bstrut\\
    \hline
    \multirow{2}{*}{\dec{101}} &
        \textlangle du\v{z}o \v{z}akumsgi gasdi\textrangle & \Tstrut\\
        & \textlangle gadu\v{z}o \v{z}akumsgi gasdi\textrangle & \Bstrut\\
    \hline
    \multirow{2}{*}{\dec{102}} &
        \textlangle du\v{z}o \v{z}akumsgi dosde\textrangle & \Tstrut\\
        & \textlangle du\v{z}o \v{z}akumsgi dosde\textrangle & \Bstrut\\
    \hline
    \multicolumn{3}{|c|}{\dots} \TBstrut\\
    \hline

    \multirow{2}{*}{\dec{110}} &
        \textlangle du\v{z}o kumsgi\textrangle & \Tstrut\\
        & \textlangle gadu\v{z}o gakumsgi\textrangle & \Bstrut\\
    \hline
    \multirow{2}{*}{\dec{120}} &
        \textlangle du\v{z}o dokumsgi\textrangle & \Tstrut\\
        & \textlangle gadu\v{z}o dokumsgi\textrangle & \Bstrut\\
    \hline
    \multicolumn{3}{|c|}{\dots} \TBstrut\\
    \hline

    \dec{200} &
        \textlangle dodu\v{z}o\textrangle & \TBstrut\\
    \hline
    \dec{300} &
        \textlangle \v{s}iduzo\textrangle & \TBstrut\\
    \hline
    \dec{400} &
        \textlangle bedu\v{z}o\textrangle & \TBstrut\\
    \hline
    \dec{500} &
        \textlangle cidu\v{z}o\textrangle & \TBstrut\\
    \hline
    \dec{600} &
        \textlangle \v{s}aduzo\textrangle & \TBstrut\\
    \hline
    \dec{700} &
        \textlangle kidu\v{z}o\textrangle & \TBstrut\\
    \hline
    \dec{800} &
        \textlangle \v{z}iduzo\textrangle & \TBstrut\\
    \hline
    \dec{900} &
        \textlangle zedu\v{z}o\textrangle & \TBstrut\\
    \hline
\end{longtable}

\subsubsection{Inanimate}
Numerals modifying \acrlong{inan} nouns are base 12.

\begin{longtable}[l]{|c|c|c|c|}
    \hline
    \textbf{Dozenal} &
        \textbf{Decimal} &
        \textbf{Sumsgiwa} & \TBstrut\\
    \hline
    \endhead

    \doz{0} & \dec{0} &
        \textlangle kamda\textrangle & \TBstrut\\
    \hline
    \doz{1} & \dec{1} &
        \textlangle \v{s}udo\textrangle & \TBstrut\\
    \hline
    \doz{2} & \dec{2} &
        \textlangle ki\v{s}e\textrangle & \TBstrut\\
    \hline
    \doz{3} & \dec{3} &
        \textlangle \v{c}e\v{s}in\textrangle & \TBstrut\\
    \hline
    \doz{4} & \dec{4} &
        \textlangle sgonti\textrangle & \TBstrut\\
    \hline
    \doz{5} & \dec{5} &
        \textlangle \v{z}ago\textrangle & \TBstrut\\
    \hline
    \doz{6} & \dec{6} &
        \textlangle pi\v{z}u\textrangle & \TBstrut\\
    \hline
    \doz{7} & \dec{7} &
        \textlangle \v{s}deke\textrangle & \TBstrut\\
    \hline
    \doz{8} & \dec{8} &
        \textlangle semdum\textrangle & \TBstrut\\
    \hline
    \doz{9} & \dec{9} &
        \textlangle cu\v{s}e\textrangle & \TBstrut\\
    \hline
    \doz{$\chi$} & \dec{10} &
        \textlangle \v{s}odu\textrangle & \TBstrut\\
    \hline
    \doz{$\xi$} & \dec{11} &
        \textlangle \v{z}udan\textrangle & \TBstrut\\
    \hline
    \multirow{2}{*}{\doz{10}} & \multirow{2}{*}{\dec{12}} &
        \textlangle bo\v{s}a\textrangle & \Tstrut\\
        & & \textlangle \v{s}ubosa\textrangle & \Bstrut\\
    \hline
    
    \multirow{2}{*}{\doz{11}} & \multirow{2}{*}{\dec{13}} &
        \textlangle bo\v{s}a \v{s}udo\textrangle & \Tstrut\\
        & & \textlangle \v{s}ubosa \v{s}udo\textrangle & \Bstrut\\
    \hline
    \multirow{2}{*}{\doz{12}} & \multirow{2}{*}{\dec{14}} &
        \textlangle bo\v{s}a ki\v{s}e\textrangle & \Tstrut\\
        & & \textlangle \v{s}ubosa ki\v{s}e\textrangle & \Bstrut\\
    \hline
    \multicolumn{4}{|c|}{\dots} \TBstrut\\
    \hline
    
    \doz{20} & \dec{24} &
        \textlangle kibo\v{s}a\textrangle & \TBstrut\\
    \hline
    \doz{21} & \dec{25} &
        \textlangle kibo\v{s}a \v{s}udo\textrangle & \TBstrut\\
    \hline
    \multicolumn{4}{|c|}{\dots} \TBstrut\\
    \hline
    
    \doz{30} & \dec{36} &
        \textlangle \v{c}ebo\v{s}a\textrangle & \TBstrut\\
    \hline
    \doz{40} & \dec{48} &
        \textlangle sgobo\v{s}a\textrangle & \TBstrut\\
    \hline
    \doz{50} & \dec{60} &
        \textlangle \v{z}abosa\textrangle & \TBstrut\\
    \hline
    \doz{60} & \dec{72} &
        \textlangle pibo\v{s}a\textrangle & \TBstrut\\
    \hline
    \doz{70} & \dec{84} &
        \textlangle \v{s}debo\v{s}a\textrangle & \TBstrut\\
    \hline
    \doz{80} & \dec{96} &
        \textlangle sempo\v{s}a\textrangle & \TBstrut\\
    \hline
    \doz{90} & \dec{108} &
        \textlangle cubosa\textrangle & \TBstrut\\
    \hline
    \doz{$\chi$0} & \dec{120} &
        \textlangle \v{s}obosa\textrangle & \TBstrut\\
    \hline
    \doz{$\xi$0} & \dec{132} &
        \textlangle \v{z}ubosa\textrangle & \TBstrut\\
    \hline
    \multicolumn{4}{|c|}{\dots} \TBstrut\\
    \hline
    
    \multirow{2}{*}{\doz{100}} & \multirow{2}{*}{\dec{144}} &
        \textlangle ko\v{s}im\textrangle & \Tstrut\\
        & & \textlangle \v{s}uko\v{s}im\textrangle & \Bstrut\\
    \hline
    \multirow{2}{*}{\doz{101}} & \multirow{2}{*}{\dec{145}} &
        \textlangle ko\v{s}im kamposa \v{s}udo\textrangle & \Tstrut\\
        & & \textlangle \v{s}uko\v{s}im kamposa \v{s}udo\textrangle & \Bstrut\\
    \hline
    \multirow{2}{*}{\doz{102}} & \multirow{2}{*}{\dec{146}} &
        \textlangle ko\v{s}im kamposa ki\v{s}e\textrangle & \Tstrut\\
        & & \textlangle \v{s}uko\v{s}im kamposa ki\v{s}e\textrangle & \Bstrut\\
    \hline
    \multicolumn{4}{|c|}{\dots} \TBstrut\\
    \hline

    \multirow{2}{*}{\doz{110}} & \multirow{2}{*}{\dec{156}} &
        \textlangle ko\v{s}im bo\v{s}a\textrangle & \Tstrut\\
        & & \textlangle \v{s}uko\v{s}im \v{s}ubosa\textrangle & \Bstrut\\
    \hline
    \multirow{2}{*}{\doz{120}} & \multirow{2}{*}{\dec{168}} &
        \textlangle ko\v{s}im kibo\v{s}a\textrangle & \Tstrut\\
        & & \textlangle \v{s}uko\v{s}im kibo\v{s}a\textrangle & \Bstrut\\
    \hline
    \multicolumn{4}{|c|}{\dots} \TBstrut\\
    \hline
    
    \doz{200} & \dec{288} &
        \textlangle kigosim\textrangle & \TBstrut\\
    \hline
    \doz{300} & \dec{432} &
        \textlangle \v{c}eko\v{s}im\textrangle & \TBstrut\\
    \hline
    \doz{400} & \dec{576} &
        \textlangle sgokosim\textrangle & \TBstrut\\
    \hline
    \doz{500} & \dec{720} &
        \textlangle \v{z}akosim\textrangle & \TBstrut\\
    \hline
    \doz{600} & \dec{864} &
        \textlangle piko\v{s}im\textrangle & \TBstrut\\
    \hline
    \doz{700} & \dec{1008} &
        \textlangle \v{s}degesim\textrangle & \TBstrut\\
    \hline
    \doz{800} & \dec{1152} &
        \textlangle semgo\v{s}im\textrangle & \TBstrut\\
    \hline
    \doz{900} & \dec{1296} &
        \textlangle cukosim\textrangle & \TBstrut\\
    \hline
    \doz{$\chi$00} & \dec{1440} &
        \textlangle \v{s}oko\v{s}im\textrangle & \TBstrut\\
    \hline
    \doz{$\xi$00} & \dec{1584} &
        \textlangle \v{z}ukosim\textrangle & \TBstrut\\
    \hline
\end{longtable}

\subsection{D-Quantifiers}
D-quantifiers agree with nouns in gender.

\begin{tabular}{|l|c|c|c|}
    \hline
    & \Glossfull{hg} &
        \Glossfull{an} &
        \Glossfull{inan} \TBstrut\\
    \hline

    \multirow{2}{*}{none/zero} &
        \textlangle zosi\textrangle &
        \textlangle sodi\textrangle &
        \textlangle \v{s}ugi\textrangle \Tstrut\\
        & & & \Bstrut\\
    \hline

    \multirow{2}{*}{some/a few} &
        \textlangle diñ\v{s}u\textrangle &
        \textlangle po\v{s}u\textrangle &
        \textlangle \v{c}ogi\textrangle \Tstrut\\
        & & & \Bstrut\\
    \hline

    \multirow{2}{*}{many/most} &
        \textlangle \v{s}esbun\textrangle &
        \textlangle sbeso\textrangle &
        \textlangle \v{s}base\textrangle \Tstrut\\
        & & & \Bstrut\\
    \hline

    \multirow{2}{*}{all/every} &
        \textlangle \v{z}i\.{z}e\textrangle &
        \textlangle ka\v{s}o\textrangle &
        \textlangle \.{z}e\v{s}o\textrangle \Tstrut\\
        & & & \Bstrut\\
    \hline
\end{tabular}

\subsection{A-Quantifiers}
Explicit A-quantifiers are not applicable to \acrlong{pfv} and \acrlong{prog} aspects.

\begin{tabular}{|m{8em}|l|c|c|}
    \hline
    \textbf{Meaning} &
        \textbf{\Acrshort{tam}} &
        \textbf{Adverb} &
        \textbf{} \TBstrut\\
    \hline
    usually/often &
        \Acrlong{pos} \Acrlong{hab} &
        & \TBstrut\\
    \hline
    usually doesn't/\newline only rarely &
        \Acrlong{neg} \Acrlong{hab} &
        & \TBstrut\\
    \hline
    always &
        \Acrlong{pos} \Acrlong{hab} &
        \textlangle siyu\v{z}i\textrangle & \TBstrut\\
    \hline
    never &
        \Acrlong{pos} \Acrlong{hab} &
        \textlangle ke\v{z}in\textrangle & \TBstrut\\
    \hline
    Specified number\newline of repetitions &
        \Acrlong{pos} \Acrlong{iter} &
        \textlangle \v{z}edun\textrangle~+ \Inan~numeral & \TBstrut\\
    \hline
\end{tabular}
\section{Lexicon}
\url{https://docs.google.com/spreadsheets/d/1W3hW0CICs0Jzt5hmA5ESsOZLXcQeZ_dRlPPZJA39RSg/edit?usp=sharing}

\begin{multicols}{2}
\begin{hangparas}{1em}{1}

\entry{pi\v{s}e}{piSe}{adj.}{sad. }

\entry{pode}{pode}{an. n.}{cat. }

\entry{bi\v{s}e}{biSe}{inan. n.}{table. }

\entry{bi\v{s}en}{biSen}{adv.}{quickly. }

\entry{ti\~{n}zim}{tiNzim}{inan. n.}{flower. }

\entry{tun\v{z}u}{tunZu}{adj.}{happy. }

\entry{te\v{z}u\~{n}}{teZuN}{inan. n.}{everywhere. }

\entry{duga\v{s}a}{dugaSa}{inan. n.}{rainwater. }

\entry{duzo}{duzo}{adj.}{wild, barbaric. }

\entry{du\v{s}am}{duSam}{h. n.}{person, human. }

\entry{dem}{dem}{adp.}{inside. }

\entry{desige}{desige}{adv.}{again. }

\entry{dese}{dese}{adv.}{probably, likely. }

\entry{do}{do}{adp.}{toward. }

\entry{do\v{s}bi}{doSbi}{v.}{to stop, to finish. }

\entry{do\.{z}e\v{s}o}{do\t{dz}eSo}{v.}{to jump. }

\entry{ki\v{s}i}{kiSi}{inan. n.}{light. }

\entry{ki\v{z}e}{kiZe}{h. n.}{boy. }

\entry{ku}{ku}{v.}{to go. }

\entry{ku\v{s}e}{kuSe}{adv.}{slowly. }

\entry{ke\~{n}}{keN}{adv.}{now. }

\entry{ke\v{s}de}{keSde}{inan. n.}{there (\acrlong{med}, near listener). Can refere to specifc place or anywhere near speaker depending on context. In contrast to \textlangle sbi\v{z}a\textrangle  and \textlangle \v{z}anti\textrangle .}

\entry{ke\v{s}gi}{keSgi}{v.}{to roll. }

\entry{ke\v{s}e}{keSe}{adv.}{in the past, a long time ago. }

\entry{kon}{kon}{v.}{to eat. }

\entry{ginsbu}{ginsbu}{v.}{to rise. }

\entry{getu}{getu}{adv.}{soon. }

\entry{ga\v{z}esu}{gaZesu}{an. n.}{animal. }

\entry{\~{n}izu}{Nizu}{v.}{walk (of an animal). }

\entry{\~{n}eke}{Neke}{an. n.}{mist, fog. }

\entry{s-}{s}{aff.}{\acrlong{ind} verb. }

\entry{sbi\v{z}a}{sbiZa}{inan. n.}{here (\acrlong{prox}). Can refere to specifc place or anywhere near speaker depending on context. In contrast to \textlangle ke\v{s}de\textrangle  and \textlangle \v{z}anti\textrangle .}

\entry{sdandi}{sdandi}{h. n.}{girl. }

\entry{sinzi}{sinzi}{v.}{to come. }

\entry{sukin}{sukin}{adj.}{bright. }

\entry{so\v{z}a\v{s}be}{soZaSbe}{v.}{to cooperate. }

\entry{-\v{s}bi}{Sbi}{aff.}{verb with \acrlong{inan} \gls{subject}. }

\entry{\v{s}bedim}{Sbedim}{adv.}{tomorrow. }

\entry{\v{s}ikunso}{Sikunso}{inan. n.}{ball. }

\entry{\v{s}in\v{z}e}{SinZe}{adj.}{very young; child. }

\entry{\v{s}umzan}{Sumzan}{an. n.}{dog. }

\entry{\v{s}ede}{Sede}{h. n.}{child. }

\entry{\v{s}emdin}{Semdin}{h. n.}{baby. }

\entry{\v{s}o}{So}{v.}{to look, to see. }

\entry{\v{s}onyu}{Son\t{dZ}u}{v.}{to fall. }

\entry{\v{s}o\v{z}idu}{SoZidu}{v.}{to play. }

\entry{-\v{z}u}{Zu}{aff.}{verb with \acrlong{an} \gls{subject}. }

\entry{-\v{z}u}{Zu}{aff.}{\acrlong{an} adjective. }

\entry{\v{z}u\v{s}o}{ZuSo}{adj.}{pretty, beautiful, handsome. }

\entry{\v{z}ege}{Zege}{inan. n.}{sky. }

\entry{\v{z}ojem}{Zo\t{dZ}em}{inan. n.}{sun. }

\entry{\v{z}am}{Zam}{adp.}{to (a location). }

\entry{\v{z}anti}{Zanti}{inan. n.}{there (\acrlong{dist}, far from both listener and speaker). Can refer to specific place or anywhere near speaker depending on context. In contrast to \textlangle sbi\v{z}a\textrangle and \textlangle ke\v{s}de\textrangle .}

\entry{ci}{\t{ts}i}{adp.}{away from. }

\entry{\.{z}e}{\t{dz}e}{v.}{to give. }

\entry{\v{c}o}{\t{tS}o}{adp.}{surrounded by, inside. }

\entry{\v{c}o\v{s}da\v{z}e}{\t{tS}oSdaZe}{v.}{to shout, to yell. }

\entry{\v{c}a-}{\t{tS}a}{aff.}{\acrlong{nom} adjective. }

\entry{ja-}{\t{dZ}a}{aff.}{\acrlong{nom} noun. }

\entry{i-}{i}{aff.}{\acrlong{pfv} verb. }

\entry{a-}{a}{aff.}{\acrlong{prog} verb. }


\end{hangparas}
\end{multicols}
\section{Translations}

\subsection{Conlang Syntax Test Cases}
\url{https://web.archive.org/web/20130603121930/http://fiziwig.com/conlang/syntax_tests.html}

\begingroup
\lingset{
    lingstyle=defaultgloss,
    exskip=-10pt,
    exnoformat=(CSTC.X)
}
\excnt=1

\ex~[glstyle=nlevel]
\begingl
\glpreamble \sumsgiwasub{sini\.{z}esba\v{z}odi\v{z}ojem\v{s}aki\v{s}i} \\ sini\.{z}esba\v{z}o di\v{z}ojem \v{s}aki\v{s}i
\endpreamble
s-[{\Ind}-]@
ini-[{\Hab}-]@
\.{z}e[give]@
-sba[-{\Inan}]@
-\v{z}o[-{\Inan}]
di-[{\Erg}-]@
\v{z}ojem[sun]
\v{s}a-[{\Acc}-]@
ki\v{s}i[light]
\glft "The sun shines." (lit. "The sun gives light.")
\endgl
\xe

\ex~[glstyle=nlevel]
\begingl 
\glpreamble \SG{\v{s}gezusini\.{z}esba\v{z}odi\v{z}ojem\v{s}aki\v{s}i} \\ \v{s}gezu sini\.{z}esba\v{z}o di\v{z}ojem \v{s}aki\v{s}i
\endpreamble
\v{s}gezu[{\Yn}]
s-[{\Hab}-]@
ini-[{\Prog}-]@
\.{z}e[give]@
-sba[-{\Inan}]@
-\v{z}o[-{\Inan}]
di-[{\Erg}-]@
\v{z}ojem[sun]
\v{s}a-[{\Acc}-]@
ki\v{s}i[light]
\glft "Does the sun shine?" (lit. "Does the sun give light?")
\endgl
\xe

\ex~[glstyle=nlevel]
\begingl
\glpreamble \SG{se\.{z}esba\v{z}odi\v{z}ojem\v{s}aki\v{s}i} \\ se\.{z}esba\v{z}o di\v{z}ojem \v{s}aki\v{s}i
\endpreamble
s-[{\Ind}-]@
e-[{\Pfv}-]@
\.{z}e[give]@
-sba[-{\Inan}]@
-\v{z}o[-{\Inan}]
di-[{\Erg}-]@
\v{z}ojem[sun]
\v{s}a-[{\Acc}-]@
ki\v{s}i[light]
\glft "The sun shone." (lit. "The sun gave light.")
\endgl
\xe

\ex~[glstyle=nlevel]
\begingl
\glpreamble \SG{ye\.{z}esba\v{z}odi\v{z}ojem\v{s}aki\v{s}i} \\ ye\.{z}esba\v{z}o di\v{z}ojem \v{s}aki\v{s}i
\endpreamble
y-[{\Sjv}-]@
e-[{\Pfv}-]@
\.{z}e[give]@
-sba[-{\Inan}]@
-\v{z}o[-{\Inan}]
di-[{\Erg}-]@
\v{z}ojem[sun]
\v{s}a-[{\Acc}-]@
ki\v{s}i[light]
\glft "The sun will shine." (lit. "The sun will give light.")
\endgl
\xe

\ex~[glstyle=nlevel]
\begingl
\glpreamble \SG{sa\.{z}esba\v{z}odi\v{z}ojem\v{s}aki\v{s}i} \\ sa\.{z}esba\v{z}o di\v{z}ojem \v{s}aki\v{s}i
\endpreamble
s-[{\Ind}-]@
a-[{\Prog}-]@
\.{z}e[give]@
-sba[-{\Inan}]@
-\v{z}o[-{\Inan}]
di-[{\Erg}-]@
\v{z}ojem[sun]
\v{s}a-[{\Acc}-]@
ki\v{s}i[light]
\glft "The sun has been shining." (lit. "The sun has been giving light.")
\endgl
\xe

\ex~[glstyle=nlevel]
\begingl
\glpreamble \SG{sa\.{z}esba\v{z}onadesigedi\v{z}ojem\v{s}aki\v{s}i} \\ sa\.{z}esba\v{z}o nadesige di\v{z}ojem \v{s}aki\v{s}i
\endpreamble
s-[{\Ind}-]@
a-[{\Prog}-]@
\.{z}e[give]@
-sba[-{\Inan}]@
-\v{z}o[-{\Inan}]
na-[{\Prog}-]@
desige[again]
di-[{\Erg}-]@
\v{z}ojem[sun]
\v{s}a-[{\Acc}-]@
ki\v{s}i[light]
\glft "The sun is shining again." (lit. "The sun is giving light again.")
\endgl
\xe

\ex~[glstyle=nlevel]
\begingl
\glpreamble \SG{se\.{z}esba\v{z}o\v{s}bedimdi\v{z}ojem\v{s}aki\v{s}i} \\ se\.{z}esba\v{z}o \v{s}bedim di\v{z}ojem \v{s}aki\v{s}i
\endpreamble
s-[{\Ind}-]@
e-[{\Pfv}-]@
\.{z}e[give]@
-sba[-{\Inan}]@
-\v{z}o[-{\Inan}]
Ø-[{\Pfv}-]@
\v{s}bedim[tomorrow]
di-[{\Erg}-]@
\v{z}ojem[sun]
\v{s}a-[{\Acc}-]@
ki\v{s}i[light]
\glft "The sun will shine tomorrow." (lit. "The sun will give light tomorrow.")
\endgl
\xe

\ex~[glstyle=nlevel]
\begingl
\glpreamble \SG{sini\.{z}esba\v{z}onasukindi\v{z}ojem\v{s}aki\v{s}i} \\ sini\.{z}esba\v{z}o nasukin di\v{z}ojem \v{s}aki\v{s}i
\endpreamble
s-[{\Ind}-]@
ini-[{\Hab}-]@
\.{z}e[give]@
-sba[-{\Inan}]@
-\v{z}o[-{\Inan}]
na-[{\Prog}-]@
sukin[bright]
di-[{\Erg}-]@
\v{z}ojem[sun]
\v{s}a-[{\Acc}-]@
ki\v{s}i[light]
\glft "The sun shines brightly." (lit. "The sun gives light brightly.")
\endgl
\xe

\ex~[glstyle=nlevel]
\begingl 
\glpreamble \SG{sini\.{z}esba\v{z}otisukin\v{s}bidi\v{z}ojem\v{s}aki\v{s}i} \\ sini\.{z}esba\v{z}o tisukin\v{s}bi di\v{z}ojem \v{s}aki\v{s}i
\endpreamble
s-[{\Ind}-]@
ini-[{\Hab}-]@
\.{z}e[give]@
-sba[-{\Inan}]@
-\v{z}o[-{\Inan}]
ti-[{\Erg}-]@
sukin[bright]@
-\v{s}bi[-{\Inan}]
di-[{\Erg}-]@
\v{z}ojem[sun]
\v{s}a-[{\Acc}-]@
ki\v{s}i[light]
\glft "The bright sun shines." (lit. "The bright sun gives light.")
\endgl
\xe

\ex~[glstyle=nlevel]
\begingl
\glpreamble \SG{sasinzi\v{s}binakeñja\v{z}ojem} \\ sasinzi\v{s}bi nakeñ ja\v{z}ojem
\endpreamble
s-[{\Ind}-]@
a-[{\Prog}-]@
sinzi[come]@
-\v{s}bi[-{\Inan}]
na-[{\Prog}-]@
keñ[now]
ja-[{\Nom}-]@
\v{z}ojem[sun]
\glft "The sun is rising now." (lit. "The sun is coming now.")
\endgl
\xe

\ex~[glstyle=nlevel]
\begingl 
\glpreamble \SG{\v{c}o\v{s}da\v{z}epojadu\v{s}am\v{z}i\.{z}e} \\ \v{c}o\v{s}da\v{z}epo jadu\v{s}am \v{z}i\.{z}e
\endpreamble
Ø-[{\Ind}-]@
Ø-[{\Pfv}-]@
\v{c}o\v{s}da\v{z}e[shout]@
-po[-{\Hg}]
ja-[{\Nom}-]@
du\v{s}am[person]
\v{z}i\.{z}e[all.{\Hg}]
\glft "All the people shouted."
\endgl
\xe

\ex~[glstyle=nlevel]
\begingl
\glpreamble \SG{\v{c}o\v{s}da\v{z}epojadu\v{s}amdiñ\v{s}u} \\ \v{c}o\v{s}da\v{z}epo jadu\v{s}am diñ\v{s}u
\endpreamble
Ø-[{\Ind}-]@
Ø-[{\Pfv}-]@
\v{c}o\v{s}da\v{z}e[shout]
-po[-{\Hg}]
ja-[{\Nom}-]@
du\v{s}am[person]
diñ\v{s}u[some.{\Hg}]
\glft "Some of the people shouted."
\endgl
\xe

\ex~[glstyle=nlevel]
\begingl 
\glpreamble \SG{si\v{c}o\v{s}da\v{z}epo\v{z}edunki\v{s}ejadu\v{s}am\v{s}esbun} \\ si\v{c}o\v{s}da\v{z}epo \v{z}edun ki\v{s}e jadu\v{s}am \v{s}esbun
\endpreamble
s-[{\Ind}-]@
i-[{\Iter}-]@
\v{c}o\v{s}da\v{z}e[shout]@
-po[-{\Hg}]
\v{z}edun[times]
ki\v{s}e[two.{\Inan}]
ja-[{\Nom}-]@
du\v{s}am[person]
\v{s}esbun[many.{\Hg}]
\glft "Many of the people shouted twice."
\endgl
\xe

\ex~[glstyle=nlevel]
\begingl
\glpreamble \SG{sini\v{c}o\v{s}da\v{z}epojadu\v{s}am\v{c}atun\v{z}upo} \\ sini\v{c}o\v{s}da\v{z}epo jadu\v{s}am \v{c}atun\v{z}upo
\endpreamble
s-[{\Ind}-]@
ini-[{\Hab}-]@
\v{c}o\v{s}da\v{z}e[shout]@
-po[-{\Hg}]
ja-[{\Nom}-]@
du\v{s}am[person]
\v{c}a-[{\Nom}-]@
tun\v{z}u[happy]@
-po[-{\Hg}]
\glft "Happy people often shout."
\endgl
\xe

\ex~[glstyle=nlevel]
\begingl 
\glpreamble \SG{do\.{z}e\v{s}o\v{z}udo\v{z}egejapode\v{c}a\v{s}in\v{z}e\v{z}u} \\ do\.{z}e\v{s}o\v{z}u do\v{z}ege japode \v{c}a\v{s}in\v{z}e\v{z}u
\endpreamble
Ø-[{\Ind}-]@
Ø-[{\Pfv}-]@
do\.{z}e\v{s}o[jump]@
-\v{z}u[-{\An}]
do-[toward-]@
\v{z}ege[sky]
ja-[{\Nom}-]@
pode[cat]
\v{c}a-[{\Nom}-]@
\v{s}in\v{z}e[very\_young]@
-\v{z}u[-{\An}]
\glft "The kitten jumped up."
\endgl
\xe

\ex~[glstyle=nlevel]
\begingl 
\glpreamble \SG{do\.{z}e\v{s}o\v{z}u\v{z}ambi\v{s}ejapode\v{c}a\v{s}in\v{z}e\v{z}u} \\ do\.{z}e\v{s}o\v{z}u \v{z}ambi\v{s}e japode \v{c}a\v{s}in\v{z}e\v{z}u
\endpreamble
Ø-[{\Ind}-]@
Ø-[{\Pfv}-]@
do\.{z}e\v{s}o[jump]@
-\v{z}u[-{\An}]
\v{z}am-[to-]@
bi\v{s}e[table]
ja-[{\Nom}-]@
pode[cat]
\v{c}a-[{\Nom}-]@
\v{s}in\v{z}e[very\_young]@
-\v{z}u[-{\An}]
\glft "The kitten jumped onto the table."
\endgl
\xe

\ex~[glstyle=nlevel]
\begingl 
\glpreamble \SG{ñizu\v{z}udo\v{z}antijapode\v{c}a\v{s}in\v{z}e\v{z}uyañi} \\ ñizu\v{z}u do\v{z}anti japode \v{c}a\v{s}in\v{z}e\v{z}u yañi
\endpreamble
Ø-[{\Ind}-]@
Ø-[{\Pfv}-]@
ñizu[walk]@
-\v{z}u[-{\An}]
do-[toward-]@
\v{z}anti[there\_{\Dist}]
ja-[{\Nom}-]@
pode[cat]
\v{c}a-[{\Nom}-]@
\v{s}in\v{z}e[very\_young]@
-\v{z}u[-{\An}]
yañi[{\Fex}.{\Hg}.{\Gen}]
\glft "My little kitten walked away."
\endgl
\xe

\ex~[glstyle=nlevel]
\begingl 
\glpreamble \SG{sa\v{s}onyu\v{s}bijaduga\v{s}a} \\ sa\v{s}onyu\v{s}bi jaduga\v{s}a
\endpreamble
s-[{\Ind}-]@
a-[{\Prog}-]@
\v{s}onyu[fall]@
-\v{s}bi[-{\Inan}]
ja-[{\Nom}-]@
duga\v{s}a[rainwater]
\glft "It's raining." (lit. "Rainwater is falling.")
\endgl
\xe

\ex~[glstyle=nlevel]
\begingl 
\glpreamble \SG{\v{s}onyu\v{s}bijaduga\v{s}a} \\ \v{s}onyu\v{s}bi jaduga\v{s}a
\endpreamble
Ø-[{\Ind}-]@
Ø-[{\Pfv}-]@
\v{s}onyu[fall]@
-\v{s}bi[-{\Inan}]
ja-[{\Nom}-]@
duga\v{s}a[rainwater]
\glft "The rain came down." (lit. "Rainwater fell.")
\endgl
\xe

\ex~[glstyle=nlevel]
\begingl
\glpreamble \SG{sa\v{s}o\v{z}idu\v{z}u\v{c}oduga\v{s}ajapode\v{c}a\v{s}in\v{z}e\v{z}u} \\ sa\v{s}o\v{z}idu\v{z}u \v{c}oduga\v{s}a japode \v{c}a\v{s}in\v{z}e\v{z}u
\endpreamble
s-[{\Ind}-]@
a-[{\Prog}-]@
\v{s}o\v{z}idu[play]@
-\v{z}u[-{\An}]
\v{c}o-[surrounded\_by-]@
duga\v{s}a[rainwater]
ja-[{\Nom}-]@
pode[cat]
\v{c}a[{\Nom}-]@
\v{s}in\v{z}e[very\_young]@
-\v{z}u[-{\An}]
\glft "The kitten is playing in the rain."
\endgl
\xe

\ex~[glstyle=nlevel]
\begingl 
\glpreamble \SG{do\v{s}bi\v{s}onyu\v{s}bijaduga\v{s}a} \\ do\v{s}bi-\v{s}onyu\v{s}bi jaduga\v{s}a
\endpreamble
Ø-[{\Ind}-]@
Ø-[{\Pfv}-]@
do\v{s}bi-[stop-]@
\v{s}onyu[fall]@
-\v{s}bi[-{\Inan}]
ja-[{\Nom}-]@
duga\v{s}a[rainwater]
\glft "The rain has stopped."
\endgl
\xe

\ex~[glstyle=nlevel]
\begingl 
\glpreamble \SG{do\v{s}bi\v{s}onyu\v{s}bigetujaduga\v{s}a} \\ do\v{s}bi-\v{s}onyu\v{s}bi getu jaduga\v{s}a
\endpreamble
Ø-[{\Ind}-]@
Ø-[{\Pfv}-]@
do\v{s}bi-[stop-]@
\v{s}onyu[fall]@
-\v{s}bi[-{\Inan}]
Ø-[{\Pfv}-]@
getu[soon]
ja-[{\Nom}-]@
duga\v{s}a[rainwater]
\glft "Soon the rain will stop."
\endgl
\xe

\ex~[glstyle=nlevel]
\begingl
\glpreamble \SG{\v{s}dedo\v{s}bi\v{s}onyu\v{s}bigetujaduga\v{s}a} \\ \v{s}dedo\v{s}bi-\v{s}onyu\v{s}bi getu jaduga\v{s}a
\endpreamble
\v{s}d-[{\Opt}-]@
e-[{\Pfv}-]@
do\v{s}bi-[stop-]@
\v{s}onyu[fall]@
-\v{s}bi[-{\Inan}]
Ø-[{\Pfv}-]@
getu[soon]
ja-[{\Nom}-]@
duga\v{s}a[rainwater]
\glft "I hope the rain stops soon."
\endgl
\xe

\ex~[glstyle=nlevel]
\begingl
\glpreamble \SG{seke\v{s}e\v{c}osbi\v{z}ajaga\v{z}esu\v{c}aduzo\v{z}u} \\ seke\v{s}e \v{c}osbi\v{z}a jaga\v{z}esu \v{c}aduzo\v{z}u
\endpreamble
Ø-[{\Ind}-]@
se[{\Cop}]
Ø-[{\Pfv}-]@
ke\v{s}e[in\_the\_past]
\v{c}o-[inside-]@
sbi\v{z}a[here\_\ACRshort{prox}]
ja-[{\Nom}-]@
ga\v{z}esu[animal]
\v{c}a-[{\Nom}-]@
duzo[wild]@
-\v{z}u[-{\An}]
\glft "Once wild animals lived here."
\endgl
\xe

\ex~[glstyle=nlevel]
\begingl
\glpreamble \SG{sa\v{s}ojo\v{z}onaku\v{s}e\v{c}a\v{s}ate\v{z}uñ} \\ sa\v{s}ojo\v{z}o naku\v{s}e \v{c}a \v{s}ate\v{z}uñ
\endpreamble
s-[{\Ind}-]@
a-[{\Prog}-]@
\v{s}o[look]@
-jo[-{\Hg}]@
-\v{z}o[-{\Inan}]
na-[{\Prog}-]@
ku\v{s}e[slowly]
\v{c}a[{\T}.{\Hg}.{\Erg}]
\v{s}a-[{\Acc}-]@
te\v{z}uñ[everywhere]
\glft "Slowly she looked around."
\endgl
\xe

\ex~[glstyle=nlevel]
\begingl
\glpreamble \SG{sesum\v{s}u\v{c}ekupo} \\ sesum \v{s}u\v{c}ekupo
\endpreamble
sesum[\Imp]
\v{s}u\v{c}-[{\Deo}-]@
e-[{\Pfv}-]@
ku[go]@
-po[-{\Hg}]
\glft "Go away!"
\endgl
\xe

\ex~[glstyle=nlevel]
\begingl
\glpreamble \SG{\v{s}dekupo\v{z}em} \\ \v{s}dekupo \v{z}em
\endpreamble
\v{s}d-[{\Opt}-]@
e-[{\Pfv}-]@
ku[go]@
-po[-{\Hg}]
\v{z}em[{\Fin}.{\Hg}.{\Nom}]
\glft "Let's go!"
\endgl
\xe

\ex~[glstyle=nlevel]
\begingl
\glpreamble \SG{\v{s}u\v{c}ekupoji\v{s}bem} \\ \v{s}u\v{c}ekupo ji\v{s}bem
\endpreamble
\v{s}u\v{c}-[{\Deo}-]@
e-[{\Pfv}-]@
ku[go]@
-po[-{\Hg}]
ji\v{s}bem[{\Spol}.{\Hg}.{\Nom}]
\glft "You should go."
\endgl
\xe

\ex~[glstyle=nlevel]
\begingl
\glpreamble \SG{yekupotun\v{z}u} \\ yekupo tun\v{z}u
\endpreamble
y-[{\Subj}-]@
e-[{\Pfv}-]@
ku[go]@
-po[-{\Hg}]
Ø-[{\Pfv}-]@
tun\v{z}u[happily]
\glft "I will be happy to go." (lit. "I will go happily.")
\endgl
\xe

\ex~[glstyle=nlevel]
\begingl
\glpreamble \SG{yesinzipogetu\v{c}em} \\ yesinzipo getu \v{c}em
\endpreamble
y-[{\Subj}-]@
e-[{\Pfv}-]@
sinzi[come]@
-po[-{\Hg}]
Ø-[{\Pfv}-]@
getu[soon]
\v{c}em[{\T}.{\Hg}.{\Nom}]
\glft "He will arrive soon."
\endgl
\xe

\ex~[glstyle=nlevel]
\begingl
\glpreamble \SG{ke\v{s}gi\v{s}bicisbi\v{z}aja\v{s}ikunsane\v{s}emdin} \\ ke\v{s}gi\v{s}bi cisbi\v{z}a ja\v{s}ikunsa ne\v{s}emdin
\endpreamble
Ø-[{\Ind}-]@
Ø-[{\Pfv}-]@
ke\v{s}gi[roll]@
-\v{s}bi[-{\Inan}]
ci-[away\_from-]@
sbi\v{z}a[here\_{\Prox}]
ja-[{\Nom}-]@
\v{s}ikunsa[ball]
ne-[{\Gen}-]@
\v{s}emdin[baby]
\glft "The baby's ball has rolled away."
\endgl
\xe

\ex~[glstyle=nlevel]
\begingl
\glpreamble \SG{saso\v{z}a\v{s}bepojaki\v{z}e\v{s}ite} \\ saso\v{z}a\v{s}bepo jaki\v{z}e \v{s}ite
\endpreamble
s-[{\Ind}-]@
a-[{\Prog}-]@
so\v{z}a\v{s}be[cooperate]@
-po[-{\Hg}]
ja-[{\Nom}-]@
ki\v{z}e[boy]
\v{s}ite[two.{\Hg}]
\glft "The two boys are working together."
\endgl
\xe


\endgroup
\iffalse


\ex~[glstyle=nlevel]
\begingl
\glpreamble \SG{} \\ 
\endpreamble

\glft 
\endgl
\xe


\fi

\clearpage
\printglossary
\printacronyms
\printglosses

\end{document}
