\section{Phonology}

\subsection{Consonants}
{
  \newcommand{\BlankCell}{\multicolumn{2}{c|}{}}

  \begin{tabular}{|c|c|c|c|c|c|c|c|c|c|c|c|c|}
    \hline
    & \multicolumn{2}{c|}{Bilabial} &
    \multicolumn{2}{c|}{Alveolar} &
    \multicolumn{2}{c|}{Postalveolar} &
    \multicolumn{2}{c|}{Palatal} &
    \multicolumn{2}{c|}{Labiovelar} &
    \multicolumn{2}{c|}{Velar} \TBstrut\\
    \hline

    \multirow{3}{*}{Plosive} &
    \textipa{[p\super{h}]} & \textipa{[b]} &
    \textipa{[t\super{h}]} & \textipa{[d]} &
    \BlankCell &
    \BlankCell &
    \BlankCell &
    \textipa{[k\super{h}]} & \textipa{[g]} \Tstrut\\

    & \textipa{/p/} & \textipa{/b/} &
    \textipa{/t/} & \textipa{/d/} &
    \BlankCell &
    \BlankCell &
    \BlankCell &
    \textipa{/k/} & \textipa{/g/} \\

    & \textlangle p\textrangle &
    \textlangle b\textrangle &
    \textlangle t\textrangle &
    \textlangle d\textrangle &
    \BlankCell &
    \BlankCell &
    \BlankCell &
    \textlangle k\textrangle &
    \textlangle g\textrangle \Bstrut\\
    \hline

    \multirow{3}{*}{Nasal} &
    & \textipa{[m]} &
    & \textipa{[n]} &
    \BlankCell &
    \BlankCell &
    \BlankCell &
    & \textipa{[N]} \Tstrut\\

    & & \textipa{/m/} &
    & \textipa{/n/} &
    \BlankCell &
    \BlankCell &
    \BlankCell &
    & \textipa{/N/} \\

    & & \textlangle m\textrangle &
    & \textlangle n\textrangle &
    \BlankCell &
    \BlankCell &
    \BlankCell &
    & \textlangle ñ\textrangle \Bstrut\\
    \hline

    \multirow{3}{*}{Fricative} &
    \BlankCell &
    \textipa{[s]} & \textipa{[z]} &
    \textipa{[S]} & \textipa{[Z]} &
    \BlankCell &
    \BlankCell &
    \BlankCell \Tstrut\\

    & \BlankCell &
    \textipa{/s/} & \textipa{/z/} &
    \textipa{/S/} & \textipa{/Z/} &
    \BlankCell &
    \BlankCell &
    \BlankCell \\

    & \BlankCell &
    \textlangle s\textrangle &
    \textlangle z\textrangle &
    \textlangle\v{s}\textrangle &
    \textlangle \v{z}\textrangle &
    \BlankCell &
    \BlankCell &
    \BlankCell \Bstrut\\
    \hline

    \multirow{3}{*}{Approximant} &
    \BlankCell &
    \BlankCell &
    \BlankCell &
    \multicolumn{2}{c|}{\textipa{[j]}} &
    \multicolumn{2}{c|}{\textipa{[w]}} &
    \BlankCell \Tstrut\\

    & \BlankCell &
    \BlankCell &
    \BlankCell &
    \multicolumn{2}{c|}{\textipa{/j/}} &
    \multicolumn{2}{c|}{\textipa{/w/}} &
    \BlankCell \\

    & \BlankCell &
    \BlankCell &
    \BlankCell &
    \multicolumn{2}{c|}{\textlangle y\textrangle} &
    \multicolumn{2}{c|}{\textlangle w\textrangle} &
    \BlankCell \Bstrut\\
    \hline

    \multirow{3}{*}{Affricate} &
    \BlankCell &
    \textipa{[\t{ts}]} & \textipa{[\t{dz}]} &
    \textipa{[\t{tS}]} & \textipa{[\t{dZ}]} &
    \BlankCell &
    \BlankCell &
    \BlankCell \Tstrut\\

    & \BlankCell &
    \textipa{/\t{ts}/} & \textipa{/\t{dz}/} &
    \textipa{/\t{tS}/} & \textipa{/\t{dZ}/} &
    \BlankCell &
    \BlankCell &
    \BlankCell \\

    & \BlankCell &
    \textlangle c\textrangle &
    \textlangle \.{z}\textrangle &
    \textlangle \v{c}\textrangle &
    \textlangle j\textrangle &
    \BlankCell &
    \BlankCell &
    \BlankCell \Bstrut\\
    \hline
  \end{tabular}
}

\subsection{Vowels}
\begin{tabular}{|c|c|c|}
  \hline
  & Front & Back \TBstrut\\
  \hline

  \multirow{3}{3em}{Close} &
  \textipa{[i]} & \textipa{[u]} \Tstrut\\
  & \textipa{/i/} & \textipa{/u/} \\
  & \textlangle i\textrangle & \textlangle u\textrangle \Bstrut\\
  \hline

  \multirow{3}{3em}{Mid} &
  \textipa{[e $\sim$ E]} & \textipa{[o]} \Tstrut\\
  & \textipa{/e/} & \textipa{/o/} \\
  & \textlangle e\textrangle & \textlangle o\textrangle \Bstrut\\
  \hline

  \multirow{3}{3em}{Open} &
  \textipa{[a]} & \Tstrut\\
  & \textipa{/a/} & \\
  & \textlangle a\textrangle & \Bstrut\\
  \hline
\end{tabular}

\subsection{Syllable Structure}
Syllables in Sumsgiwa consist of an onset, a nucleus, and a rhyme. Most syllables are open. Syllable possibilities:

\noindent\\
CV(N)\\
C = p, b, t, d, k, g, m, n, ñ, s, z, \v{s}, \v{z}, y, w, c, \.{z}, \v{c}, j\\
V = i, u, e, o, a\\
N = m, n, ñ

\noindent\\
SPV(N)\\
S = s, \v{s}\\
P = b, d, g\\
V = i, u, e, o, a\\
N = m, n, ñ

\subsection{Stress}
Stress always falls on the penultimate syllable of the word stem and does not move when prefixes or suffixes are added.
