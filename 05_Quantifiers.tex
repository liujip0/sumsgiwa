\section{Quantifiers}

\subsection{Numerals}
There are 3 numeral systems in Sumsgiwa, each modifying a different gender of noun. The \acrlong{inan} numerals are used for counting.

Ordinal numbers are formed by prefixing the cardinal numeral with \textlangle bi-\textrangle.

\subsubsection{Human}
Numerals modifying \acrlong{hg} nouns are base 8.
{
  \newcommand{\TableRow}[3]{
    \oct{#1} & \dec{#2} &
    \textlangle #3\textrangle & \SG{#3} \TBstrut\\
    \hline
  }
  \newcommand{\DoubleTableRow}[4]{
    \multirow{2}{*}{\oct{#1}} & \multirow{2}{*}{\dec{#2}} &
    \textlangle #3\textrangle & \SG{#3} \Tstrut\\
    & & \textlangle #4\textrangle & \SG{#4} \Bstrut\\
    \hline
  }
  \newcommand{\DotRow}{
    \multicolumn{4}{|c|}{\dots} \TBstrut\\
    \hline
  }

  \begin{longtable}[l]{|c|c|c|c|}
    \hline
    \textbf{Octal} &
    \textbf{Decimal} &
    \textbf{Sumsgiwa} & \TBstrut\\
    \hline
    \endhead

    \TableRow{0}{0}{pun\v{z}on}
    \TableRow{1}{1}{\v{z}añe}
    \TableRow{2}{2}{\v{s}ite}
    \TableRow{3}{3}{\v{z}oye}
    \TableRow{4}{4}{da\v{z}i}
    \TableRow{5}{5}{kindu}
    \TableRow{6}{6}{\v{s}oku}
    \TableRow{7}{7}{pa\v{z}u}
    \DoubleTableRow{10}{8}{yesa}{\v{z}aye\v{s}a}
    \DoubleTableRow{11}{9}{yesa \v{z}añe}{\v{z}aye\v{s}a \v{z}añe}
    \DoubleTableRow{12}{10}{yesa \v{s}ite}{\v{z}ayesa \v{s}ite}
    \DotRow

    \TableRow{20}{16}{\v{s}iye\v{s}a}
    \TableRow{21}{17}{\v{s}iye\v{s}a \v{z}añe}
    \DotRow

    \TableRow{30}{24}{\v{z}oye\v{s}a}
    \TableRow{40}{32}{dayesa}
    \TableRow{50}{40}{kinyesa}
    \TableRow{60}{48}{\v{s}oye\v{s}a}
    \TableRow{70}{56}{payesa}
    \DotRow

    \DoubleTableRow{100}{64}{\v{z}emdo}{\v{z}azemdo}
    \DoubleTableRow{101}{65}{\v{z}emdo puyesa \v{z}añe}{\v{z}azemdo puyesa \v{z}añe}
    \DoubleTableRow{102}{66}{\v{z}emdo puyesa \v{s}ite}{\v{z}azemdo puyesa \v{s}ite}
    \DotRow

    \DoubleTableRow{110}{72}{\v{z}emdo yesa}{\v{z}azemdo \v{z}aye\v{s}a}
    \DoubleTableRow{120}{80}{\v{z}emdo \v{s}iye\v{s}a}{\v{z}azemdo \v{s}iye\v{s}a}
    \DotRow

    \TableRow{200}{128}{\v{s}izemdo}
    \TableRow{300}{192}{zozemdo}
    \TableRow{400}{256}{da\v{z}emdo}
    \TableRow{500}{320}{ki\v{z}emdo}
    \TableRow{600}{384}{sozemdo}
    \TableRow{700}{448}{pa\v{z}emdo}
  \end{longtable}
}

\subsubsection{Animate}
Numerals modifying \acrlong{an} nouns are base 10.

{
  \newcommand{\TableRow}[2]{
    \dec{#1} &
    \textlangle #2\textrangle & \SG{#2} \TBstrut\\
    \hline
  }
  \newcommand{\DoubleTableRow}[3]{
    \multirow{2}{*}{\dec{#1}} &
    \textlangle #2\textrangle & \SG{#2} \Tstrut\\
    & \textlangle #3\textrangle & \SG{#3} \Bstrut\\
    \hline
  }
  \newcommand{\DotRow}{
    \multicolumn{3}{|c|}{\dots} \TBstrut\\
    \hline
  }

  \begin{longtable}[l]{|c|c|c|}
    \hline
    \textbf{Decimal} &
    \textbf{Sumsgiwa} & \TBstrut\\
    \hline
    \endhead

    \TableRow{0}{\v{z}an\v{z}u}
    \TableRow{1}{gasdi}
    \TableRow{2}{dosde}
    \TableRow{3}{\v{s}ikim}
    \TableRow{4}{beku}
    \TableRow{5}{cinsi}
    \TableRow{6}{\v{s}asdum}
    \TableRow{7}{ki\v{s}ge}
    \TableRow{8}{\v{z}iso}
    \TableRow{9}{zenbe}
    \DoubleTableRow{10}{kumsgi}{gakumsgi}
    \DoubleTableRow{11}{kumsgi gasdi}{gakumsgi gasdi}
    \DoubleTableRow{12}{kumsgi dosde}{gakumsgi dosde}
    \DotRow

    \TableRow{20}{dokumsgi}
    \TableRow{21}{dokumsgi gasdi}
    \DotRow

    \TableRow{30}{\v{s}ikumsgi}
    \TableRow{40}{bekumsgi}
    \TableRow{50}{cikumsgi}
    \TableRow{60}{\v{s}asgumsi}
    \TableRow{70}{ki\v{s}gumsi}
    \TableRow{80}{\v{z}ikumsgi}
    \TableRow{90}{zekumsgi}
    \DotRow

    \DoubleTableRow{100}{du\v{z}o}{gadu\v{z}o}
    \DoubleTableRow{101}{du\v{z}o \v{z}akumsgi gasdi}{gadu\v{z}o \v{z}akumsgi gasdi}
    \DoubleTableRow{102}{du\v{z}o \v{z}akumsgi dosde}{gadu\v{z}o \v{z}akumsgi dosde}
    \DotRow

    \DoubleTableRow{110}{du\v{z}o kumsgi}{gadu\v{z}o gakumsgi}
    \DoubleTableRow{120}{du\v{z}o dokumsgi}{gadu\v{z}o dokumsgi}
    \DotRow

    \TableRow{200}{dodu\v{z}o}
    \TableRow{300}{\v{s}iduzo}
    \TableRow{400}{bedu\v{z}o}
    \TableRow{500}{cidu\v{z}o}
    \TableRow{600}{\v{s}aduzo}
    \TableRow{700}{kidu\v{z}o}
    \TableRow{800}{\v{z}iduzo}
    \TableRow{900}{zedu\v{z}o}
  \end{longtable}
}

\subsubsection{Inanimate}
Numerals modifying \acrlong{inan} nouns are base 12.

{
  \newcommand{\TableRow}[3]{
    \doz{#1} & \dec{#2} &
    \textlangle #3\textrangle & \SG{#3} \TBstrut\\
    \hline
  }
  \newcommand{\DoubleTableRow}[4]{
    \multirow{2}{*}{\doz{#1}} & \multirow{2}{*}{\dec{#2}} &
    \textlangle #3\textrangle & \SG{#3} \Tstrut\\
    & & \textlangle #4\textrangle & \SG{#4} \Bstrut\\
    \hline
  }
  \newcommand{\DotRow}{
    \multicolumn{4}{|c|}{\dots} \TBstrut\\
    \hline
  }

  \begin{longtable}[l]{|c|c|c|c|}
    \hline
    \textbf{Dozenal} &
    \textbf{Decimal} &
    \textbf{Sumsgiwa} & \TBstrut\\
    \hline
    \endhead

    \TableRow{0}{0}{kamda}
    \TableRow{1}{1}{\v{s}udo}
    \TableRow{2}{2}{ki\v{s}e}
    \TableRow{3}{3}{\v{c}e\v{s}in}
    \TableRow{4}{4}{sgonti}
    \TableRow{5}{5}{\v{z}ago}
    \TableRow{6}{6}{pi\v{z}u}
    \TableRow{7}{7}{\v{s}deke}
    \TableRow{8}{8}{semdum}
    \TableRow{9}{9}{cu\v{s}e}
    \TableRow{$\chi$}{10}{\v{s}odu}
    \TableRow{$\xi$}{11}{\v{z}udan}
    \DoubleTableRow{10}{12}{bo\v{s}a}{\v{s}ubosa}
    \DoubleTableRow{11}{13}{bo\v{s}a \v{s}udo}{\v{s}ubosa \v{s}udo}
    \DoubleTableRow{12}{14}{bo\v{s}a ki\v{s}e}{\v{s}ubosa ki\v{s}e}
    \DotRow

    \TableRow{20}{24}{kibo\v{s}a}
    \TableRow{21}{25}{kibo\v{s}a \v{s}udo}
    \DotRow

    \TableRow{30}{36}{\v{c}ebo\v{s}a}
    \TableRow{40}{48}{sgobo\v{s}a}
    \TableRow{50}{60}{\v{z}abosa}
    \TableRow{60}{72}{pibo\v{s}a}
    \TableRow{70}{84}{\v{s}debo\v{s}a}
    \TableRow{80}{96}{sempo\v{s}a}
    \TableRow{90}{108}{cubosa}
    \TableRow{$\chi$0}{120}{\v{s}obosa}
    \TableRow{$\xi$0}{132}{\v{z}ubosa}
    \DotRow

    \DoubleTableRow{100}{144}{ko\v{s}im}{\v{s}uko\v{s}im}
    \DoubleTableRow{101}{145}{ko\v{s}im kamposa \v{s}udo}{\v{s}uko\v{s}im kamposa \v{s}udo}
    \DoubleTableRow{102}{146}{ko\v{s}im kamposa ki\v{s}e}{\v{s}uko\v{s}im kamposa ki\v{s}e}
    \DotRow

    \DoubleTableRow{110}{156}{ko\v{s}im bo\v{s}a}{\v{s}uko\v{s}im \v{s}ubosa}
    \DoubleTableRow{120}{168}{ko\v{s}im kibo\v{s}a}{\v{s}uko\v{s}im kibo\v{s}a}
    \DotRow

    \TableRow{200}{288}{kigosim}
    \TableRow{300}{432}{\v{c}eko\v{s}im}
    \TableRow{400}{576}{sgokosim}
    \TableRow{500}{720}{\v{z}akosim}
    \TableRow{600}{864}{piko\v{s}im}
    \TableRow{700}{1008}{\v{s}degesim}
    \TableRow{800}{1152}{semgo\v{s}im}
    \TableRow{900}{1296}{cukosim}
    \TableRow{$\chi$00}{1440}{\v{s}oko\v{s}im}
    \TableRow{$\xi$00}{1584}{\v{z}ukosim}
    \DotRow

    \DoubleTableRow{1000}{1728}{deze}{\v{s}udeze}
  \end{longtable}
}

\subsection{D-Quantifiers}\label{05_02_D-Quantifiers}
D-quantifiers agree with nouns in gender.

{
  \newcommand{\TableRow}[4]{
    \multirow{2}{*}{#1} &
    \textlangle #2\textrangle &
    \textlangle #3\textrangle &
    \textlangle #4\textrangle \Tstrut\\
    & \SG{#2} & \SG{#3} & \SG{#4} \Bstrut\\
    \hline
  }

  \begin{tabular}{|l|c|c|c|}
    \hline
    & \Glossfull{hg} &
    \Glossfull{an} &
    \Glossfull{inan} \TBstrut\\
    \hline

    \TableRow{none/zero}{zosi}{sodi}{\v{s}ugi}
    \TableRow{some/a few}{diñ\v{s}u}{po\v{s}u}{\v{c}ogi}
    \TableRow{many/most}{\v{s}esbun}{sbeso}{\v{s}base}
    \TableRow{each/every}{\v{z}i\.{z}e}{ka\v{s}o}{\.{z}e\v{s}o}
  \end{tabular}
}

\subsection{A-Quantifiers}
Explicit A-quantifiers are not applicable to \acrlong{pfv} and \acrlong{prog} aspects.

{
  \newcommand{\TableRow}[3]{
    #1 & #2 & \textlangle #3\textrangle & \SG{#3} \TBstrut\\
    \hline
  }

  \begin{tabular}{|m{8em}|l|c|c|}
    \hline
    \textbf{Meaning} &
    \textbf{\Acrshort{tam}} &
    \textbf{Adverb} &
    \textbf{} \TBstrut\\
    \hline

    \TableRow{usually/often}{\Acrlong{pos} \Acrlong{hab}}{tindu}
    \TableRow{usually doesn't/\newline only rarely}{\Acrlong{neg} \Acrlong{hab}}{kupi\v{s}o}
    \TableRow{always}{\Acrlong{pos} \Acrlong{hab}}{siyu\v{z}i}
    \TableRow{never}{\Acrlong{neg} \Acrlong{hab}}{ke\v{z}in}

    Specified number\newline of repetitions &
    \Acrlong{pos} \Acrlong{iter} &
    \textlangle \v{z}edun\textrangle~+ \Inan~numeral &
    \SG{\v{z}edun} \TBstrut\\
    \hline
  \end{tabular}
}
