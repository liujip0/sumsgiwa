\entry{piše}{piSe}{adj.}{sad. }

\entry{pode}{pode}{an. n.}{cat. }

\entry{biše}{biSe}{inan. n.}{table. }

\entry{tunžu}{tunZu}{adj.}{happy. }

\entry{težuñ}{teZuN}{inan. n.}{everywhere. }

\entry{dugaša}{dugaSa}{inan. n.}{rainwater. }

\entry{duzo}{duzo}{adj.}{wild, barbaric. }

\entry{dušam}{duSam}{h. n.}{person, human. }

\entry{dem}{dem}{adp.}{inside. }

\entry{desige}{desige}{adv.}{again. }

\entry{do}{do}{adp.}{toward. }

\entry{došbi}{doSbi}{v.}{to stop, to finish. }

\entry{dożešo}{do\t{dz}eSo}{v.}{to jump. }

\entry{kiši}{kiSi}{inan. n.}{light. }

\entry{ku}{ku}{v.}{to go. }

\entry{kuše}{kuSe}{adv.}{slowly. }

\entry{keñ}{keN}{adv.}{now. }

\entry{kešde}{keSde}{inan. n.}{there (\acrlong{med}, near listener). Can refere to specifc place or anywhere near speaker depending on context. In contrast to ⟨sbiža⟩ and ⟨žanti⟩.}

\entry{kešgi}{keSgi}{v.}{to roll. }

\entry{keše}{keSe}{adv.}{in the past, a long time ago. }

\entry{ginsbu}{ginsbu}{v.}{to rise. }

\entry{getu}{getu}{adv.}{soon. }

\entry{gažesu}{gaZesu}{an. n.}{animal. }

\entry{ñizu}{Nizu}{v.}{walk (of an animal). }

\entry{s-}{s}{aff.}{\acrlong{ind} verb. }

\entry{sbiža}{sbiZa}{inan. n.}{here (\acrlong{prox}). Can refere to specifc place or anywhere near speaker depending on context. In contrast to ⟨kešde⟩ and ⟨žanti⟩.}

\entry{sinzi}{sinzi}{v.}{to come. }

\entry{sukin}{sukin}{adj.}{bright. }

\entry{-šbi}{Sbi}{aff.}{verb with \acrlong{inan} \gls{subject}. }

\entry{šbedim}{Sbedim}{adv.}{tomorrow. }

\entry{šikunso}{Sikunso}{inan. n.}{ball. }

\entry{šinže}{SinZe}{adj.}{very young; child. }

\entry{šemdin}{Semdin}{h. n.}{baby. }

\entry{šo}{So}{v.}{to look, to see. }

\entry{šonyu}{Son\t{dZ}u}{v.}{to fall. }

\entry{šožidu}{SoZidu}{v.}{to play. }

\entry{-žu}{Zu}{aff.}{verb with \acrlong{an} \gls{subject}. }

\entry{-žu}{Zu}{aff.}{\acrlong{an} adjective. }

\entry{žege}{Zege}{inan. n.}{sky. }

\entry{žojem}{Zo\t{dZ}em}{inan. n.}{sun. }

\entry{žam}{Zam}{adp.}{to (a location). }

\entry{žanti}{Zanti}{inan. n.}{there (\acrlong{dist}, far from both listener and speaker). Can refer to specific place or anywhere near speaker depending on context. In contrast to ⟨sbiža⟩ and ⟨kešde⟩.}

\entry{ci}{\t{ts}i}{adp.}{away from. }

\entry{że}{\t{dz}e}{v.}{to give. }

\entry{čo}{\t{tS}o}{adp.}{surrounded by, inside. }

\entry{čošdaže}{\t{tS}oSdaZe}{v.}{to shout, to yell. }

\entry{ča-}{\t{tS}a}{aff.}{\acrlong{nom} adjective. }

\entry{ja-}{\t{dZ}a}{aff.}{\acrlong{nom} noun. }

\entry{i-}{i}{aff.}{\acrlong{pfv} verb. }

\entry{a-}{a}{aff.}{\acrlong{prog} verb. }